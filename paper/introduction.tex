
In order to attain computer security offered by cryptography,
cryptosystems must be realized by cryptographic programs. Since modern
cryptosystems are defined on various algebraic structures,
cryptographic programs requires mathematical constructs to perform
algebraic operations on the underlying mathematical structures.
Such mathematical constructs are frequently invoked in cryptographic
programs. They are often written in low-level assembly languages and
manually optimized for efficiency. Cryptographic primitives in
cryptosystems are but sequences of algebraic operations on
mathematical structures. Security guarantees of cryptographic programs
thus depend very much on the correctness of mathematical constructs.
Security of cryptosystems could be compromised should programming
errors in mathematical constructs be exploited by attackers. 
It is therefore of utmost importance to ensure the correctness of
mathematical constructs used in cryptographic programs. In this paper,
we develop a fully certified technique to verify low-level
mathematical constructs used in the security protocol X25519
automatically.

X25519 is an Elliptic Curve Diffie-Hellman (ECDH) key exchange
protocol; it is a high-performance cryptosystem designed to 
use the secure elliptic curve Curve25519. Curve25519 is an elliptic
curve offering 128 bits of security when used with ECDH. In addition
to allowing high-speed elliptic curve arithmetic, it is easier to
implement properly, not covered by any known patents, and moreover
less susceptible to implementation pitfalls such as weak 
random-number generators. Its parameters were also selected by
easily described mathematical principles without resorting to any
random numbers or seeds. These characteristics make Curve25519 a
preferred choice among those who are leery of curves which might have
intentionally inserted
backdoors, such as curves standardized by the United States National
Institute of Standards and Technology (NIST). 
Indeed, Curve25519 is currently the
de facto alternative to the NIST P-256 curve. Consequently, X25519 has
a wide variety of applications including the default key exchange
protocol in \openssh since 2014.

Most of the computation in X25519, in trade parlance, is in a
``variable base point multiplication'', and the centerpiece of most
such software is the ``Montgomery Ladderstep''. This is usually a
large constant-time assembly language program performing the
finite-field arithmetic that implements the mathematics on Curve25519.
Should the implementation of Montgomery ladderstep be incorrect, so
would that of X25519. Obviously for all its virtues, X25519 would be
pointless if the implementation is incorrect. This may be even more
relevant in cryptography than most of engineering, because crypto is
one of the few disciplines with the concept of an omnipresent
adversary, constantly looking for the smallest edge --- and hence
eager to trigger any normally unlikely event. Revising a cryptosystem
due to rare failures potentially leading to a cryptanalysis is not
unheard of \cite{HowNTRUfail}
\todo{Insert reference in bib for http://dblp.uni-trier.de/rec/bibtex/conf/crypto/Howgrave-GrahamNPPSSW03}
Thus, it is important for security that we can show the computations
comprising the Montgomery Ladderstep or (even better) the X25519
protocol to be correct. However, such verification is not easy due
the size both of the numbers in play (255 bits and more) and of the
program itself (10,000 machine instructions upwards).

Several obstacles need be overcome in the verification of mathematical
constructs in X25519. The key exchange protocol is based on an
Abelian group induced by the elliptic curve Curve25519. The elliptic
curve is in turn defined over the Galois field $\bbfGF(2^{255} - 19)$. 
In order to compute on the Abelian group induced by Curve25519,
arithmetic computation over $\bbfGF(2^{255} - 19)$ needs to be
correct. Particularly, non-linear 255-bit multiplication modulo
$2^{255} - 19$ must be verified. Worse, commodity computing devices do
not support 255-bit arithmetic computation directly. Arithmetic over
the Galois field is implemented by sequences of 32- or 64-bit
arithmetic instructions of the underlying architectures. One has to
verify that a sequence of 32- or 64-bit arithmetic instructions indeed
performs, say, 255-bit multiplication over the finite field. And this
is but a single step in the binary operation on the group induced by
Curve25519. In order to compute the operation on Curve25519, 
the most efficient Montgomery Ladderstep contains 18 arithmetic 
operations over $\bbfGF(2^{255} - 19)$. The Ladderstep needs to be
verified to ensure security guarantees offered by Curve25519 and hence 
the ECDH key exchange protocol X25519.

In this paper, we focus on algebraic properties on low-level
implementations of mathematical constructs in cryptographic programs.
Mathematical constructs by their nature perform computation on
underlying algebraic structures. We aim to verify whether they perform
intended algebraic computation correctly. To this end, we propose the
domain specific language \mydsl for low-level implementations of
mathematical constructs. Algebraic pre- and post-conditions of
programs written in \mydsl can subsequently be specified as Hoare
triples. Such a specification is converted to single static assignment
form and then then translated into an algebraic problem (called the
modular polynomial equation entailment problem). We use the computer
algebra system \singular to solve the algebraic problem. To reduce the
size of algebraic problems and hence improve the efficiency of our
technique, program fragments irrelevant to algebraic properties are
removed by slicing. We use the proof assistant \coq to certify the
correctness of translations, as well as solutions to algebraic
problems computed by \singular.

We report case studies of verifying mathematical constructs used in
the X25519 ECDH key exchange protocol. For each arithmetic operations
(such as addition, multiplication, and square) over $\bbfGF(2^{255} - 19)$,
their low-level \qhasm implementations are converted to our domain
specific langauge \mydsl manually. We annotate each mathematical
constructs with their algebraic properties. These algebraic
specifications are then verified automatically with our technique. 
The \qhasm implementation of the Montgomery Ladderstep is formally
verified similarly.  



We have the following contributions:
\begin{itemize}
\item We propose a domain specific language for modeling low-level
  cryptographic programs over large finite fields.
\item We give a certified verification condition generator from
  programs written in our domain specific language to polynomial
  (modulo) equations over integral domains.
\item We verify arithmetic computation over a large finite field and a
  critical program (the Montgomery ladder step) automatically in a
  reasonable amount of time.  
\item To the best of our knowledge, our work is the first automatic
  and certified verification on real cryptographic programs with
  minimal human intervention.
\end{itemize}

%%% Local Variables: 
%%% mode: latex
%%% eval: (TeX-PDF-mode 1)
%%% eval: (TeX-source-correlate-mode 1)
%%% TeX-master: "certified_vcg"
%%% End: 
