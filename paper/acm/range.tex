
First define the syntax of a fragment of QF\_BV with function names taken from the standard format SMTLIB2.
In this fragment, a variable always represents a bit-vector of width $\wordsize$ (the assumed wordsize).
Let $\qExp$ and $\qPred$ respectively denote the expressions and the predicates in QF\_BV.
An expression $e \in \qExp$ can be a constant $\bvconst(n, b)$, a variiable $v \in \Var$, an addition $\bvadd(e_1, e_2)$, a subtraction $\bvsub(e_1, e_2)$, a multiplication $\bvmul(e_1, e_2)$, a concatenation $\bvconcat(e_1, e_2)$, a zero extension $\bvext(e', i)$, a left-shifting $\bvshl(e_1, e_2)$, a logical right-shifting $\bvlshr(e_1, e_2)$, or an extraction $\bvextract(e', i, j)$ where $n, i, j \in \bbfN$, $b \in \BV^n$, and $e_1, e_2, e' \in \qExp$.
A predicate $q \in \qPred$ can be $\top$, an equality $e_1 = e_2$, a less-than $\bvult(e_1, e_2)$, a less-then-or-equal $\bvule(e_1, e_2)$, a negation $\neg q'$, a conjunction $q_1 \wedge q_2$, or a disjunction $q_1 \vee q_2$ where $e_1, e_2 \in \qExp$ and $q', q_1, q_2 \in \qPred$.
An implication $q_1 \Rightarrow q_2$ is defined as $\neg q_1 \vee q_2$.

Based on the basic expressions, we define two shorthands for extracting the higher bits and the lower bits of an expression.
\[
\begin{array}{rcl}
\bvhigh(e) & \defn & \bvextract(e, 2\wordsize - 1, n) \\
\bvlow(e) & \defn & \bvextract(e, \wordsize - 1, 0) \\
\end{array}
\]
Similar to the bit-vector operations $+_\bvop$, $-_\bvop$, and $\times_\bvop$ extended with zero extension in Section~\ref{section:preliminaries}, we define the extended versions in QF\_BV for $\bullet \in \{\bvadd, \bvsub, \bvmul\}$.
For example, $\bvadd^{\#}$$(e_1,$ $e_2)$ $\defn$ $\bvadd(\bvext(e_1, \wordsize),$ $\bvext(e_2, \wordsize))$ and $\bvadd^{\#}$$(e_1,$ $e_2,$ $e_3)$ $\defn$ $\bvadd($$\bvadd($$\bvext(e_1,$ $\wordsize),$ $\bvext(e_2, \wordsize)),$ $\bvext(e_3,$ $\wordsize))$.

Let $\max(n, m)$ return the maximal number in $n$ and $m$.
Given an expression $e \in \qExp$, $\width(e)$ denotes the maximal bit-width of $e$.
\[
\begin{array}{rcl}
\width(\bvconst(n, b)) & = & n \\
\width(v) & = & w \\
\width(\bvadd(e_1, e_2)) & = & max(\width(e_1), \width(e_2))\\
\width(\bvsub(e_1, e_2)) & = & max(\width(e_1), \width(e_2))\\
\width(\bvmul(e_1, e_2)) & = & max(\width(e_1), \width(e_2))\\
\width(\bvconcat(e_1, e_2)) & = & \width(e_1) +_\bbfN \width(e_2)\\
\width(\bvext(e, i)) & = & \width(e) +_\bbfN i \\
\width(\bvshl(e_1, e_2)) & = & \width(e_1) \\
\width(\bvlshr(e_1, e_2)) & = & \width(e_1) \\
\width(\bvextract(e, i, j)) & = & i -_\bbfN j +_\bbfN 1 \\
\end{array}
\]
The expression $e$ is called \emph{well-formed} if $e$ is (1) a constant, a variable, a concatenation, a zero extension, a left-shifting, or a logical right-shifting, (2) an addition $\bvadd(e_1, e_2)$, a subtraction $\bvsub(e_1, e_2)$, or a multiplication $\bvmul(e_1, e_2)$ with $\width(e_1) = \width(e_2)$ and both $e_1$ and $e_2$ well-formed, or (3) an extraction $\bvextract(e', i, j)$ with $0 \leq j \leq i < \width(e')$ and $e'$ well-formed.
A predicate $q \in \qPred$ is well-formed if all sub-expressions are well-formed.

Let $\bvst \in \bvSt$ be a state.
Define the semantic function $\bvvalueof{e}(\bvst)$ for well-formed expressions $e$.
For a predicate $q \in \qPred$, we write $\BV^{\wordsize} \models q[\bvst]$ if $q$ evaluates to $\btt$ using the evaluation function $\bvvalueof{e}(\bvst)$ for every subexpression $e$ in $q$, using $<_\BV$ for $\bvult$, and using $\leq_\BV$ for $\bvule$.
\[
\begin{array}{rcl}
\bvvalueof{\bvconst(n, b)}(\bvst) & \defn & b \\
\bvvalueof{v}(\bvst) & \defn & \zvalueof{v}(\bvst) \\
\bvvalueof{\bvadd(e_1, e_2)}(\bvst) & \defn & \bvvalueof{e_1}(\bvst) +_\BV \bvvalueof{e_2}(\bvst) \\
\bvvalueof{\bvsub(e_1, e_2)}(\bvst) & \defn & \bvvalueof{e_1}(\bvst) -_\BV \bvvalueof{e_2}(\bvst) \\
\bvvalueof{\bvmul(e_1, e_2)}(\bvst) & \defn & \bvvalueof{e_1}(\bvst) \times_\BV \bvvalueof{e_2}(\bvst) \\
\bvvalueof{\bvconcat(e_1, e_2)}(\bvst) & \defn & \bvvalueof{e_1}(\bvst) ._\BV \bvvalueof{e_2}(\bvst) \\
\bvvalueof{\bvext(e, i)}(\bvst) & \defn & \bvvalueof{e}(\bvst) \#_\BV i \\
\bvvalueof{\bvshl(e_1, e_2)}(\bvst) & \defn & \bvvalueof{e_1}(\bvst) \ll_\BV |\bvvalueof{e_2}(\bvst)| \\
\bvvalueof{\bvlshr(e_1, e_2)}(\bvst) & \defn & \bvvalueof{e_1}(\bvst) \gg_\BV |\bvvalueof{e_2}(\bvst)| \\
\bvvalueof{\bvextract(e, i, j)}(\bvst) & \defn & \bvvalueof{e}(\bvst)[i, j] \\
\end{array}
\]

%Now we are ready to convert overflow/underflow checks and range problems to QF\_BV.
Let $q_r, q_r' \in \bvPredr$ be two range predicates and $p \in \bvProg$ a well-formed program in static single assignment form.
Both an safety check ($\BV^\wordsize \models q_r[\bvst]$ implies $\textsc{ProgSafe}(p, \bvst) = \btt$ for all $\bvst \in \bvSt$) and a range problem ($\models \hoaretriple{q_r}{p}{q_r'}$) involve only bit-vector operations and can be modeled by QF\_BV expressions.
To show that, we first define functions to transform the program $p$, the predicates $q_r$ and $q_r'$, and the safety check to QF\_BV formulas.

Define $\overline{a}$ as $v$ when the atom $a$ is a variable $v$ and otherwise $\bvconst(w, b)$ when $a$ is a constant $b$.
The function $\textsc{StmtQFBV}(s)$ (Algorithm~~\ref{algorithm:stmt-to-qfbv}) transforms a statement to a QF\_BV formula.
Recursively define the function $\textsc{ProgQFBV}$ for programs such that $\textsc{ProgQFBV}(\epsilon)$ $\defn$ $\top$ and $\textsc{ProgQFBV}$$(s;$ $p)$ $\defn$ $\textsc{StmtQFBV}($$s)$ $\land$ $\textsc{ProgQFBV}($$p)$.
Note that the formulas returned by $\textsc{StmtQFBV}(s)$ and $\textsc{ProgQFBV}(p)$ are well-formed QF\_BV formulas.
The following theorem states the soundness of the transformation from $\bvProg$ to QF\_BV formulas.

\begin{theorem}
Let $p \in \bvProg$ be a well-formed program in static single assignment form.
Then, for all $\bvst, \bvst' \in \bvSt$, $\bvst \goesto{p} \bvst'$ implies $\BV^\wordsize \models \textsc{ProgQFBV}(p)[\bvst']$.
\end{theorem}

\begin{algorithm}
  \begin{algorithmic}[1]
    \Function{StmtQFBV}{$s$}
    \Match{$s$}
      \Case{$v \leftarrow a$} \Return $v = \overline{a}$ \EndCase
      \Case{$v \leftarrow a_1 + a_2$} \Return $v = \bvadd(\overline{a_1}, \overline{a_2})$ \EndCase
      \Case{$c\ v \leftarrow a_1 + a_2$}
        \State{$r \leftarrow \bvadd^\#(\overline{a_1}, \overline{a_2})$}
        \State{\Return $c = \bvhigh(r) \wedge v = \bvlow(r)$}
      \EndCase
      \Case{$v \leftarrow a_1 + a_2 + y$}
        \State{\Return $v = \bvadd(\bvadd(\overline{a_1}, \overline{a_2}), y)$}
      \EndCase
      \Case{$c\ v \leftarrow a_1 + a_2 + y$}
        \State{$r \leftarrow \bvadd^\#(\bvadd^\#(\overline{a_1}, \overline{a_2}), y)$}
        \State{\Return $c = \bvhigh(r) \wedge v = \bvlow(r)$}
      \EndCase
      \Case{$v \leftarrow a_1 - a_2$} \Return $v = \bvsub(\overline{a_1}, \overline{a_2})$ \EndCase
      \Case{$v \leftarrow a_1 \times a_2$} \Return $v = \bvmul(\overline{a_1}, \overline{a_2})$ \EndCase
      \Case{$v_h\ v_l \leftarrow a_1 \times a_2$}
        \State{$r \leftarrow \bvmul^\#(\overline{a_1}, \overline{a_2})$}
        \State{\Return $v_h = \bvhigh(r) \wedge v_l = \bvlow(r)$}
      \EndCase
      \Case{$v \leftarrow a \ll n$} \Return $v = \bvshl(\overline{a}, \bvconst(\wordsize, n))$ \EndCase
      \Case{$v_h\ v_l \leftarrow a@n$}
        \State{$m_h \leftarrow \bvconst(\wordsize, n)$}
        \State{$m_l \leftarrow \bvconst(\wordsize, \bv^\wordsize(\wordsize - |n|))$}
        \State{\Return $v_h = \bvlshr(\overline{a}, m_h) \wedge$}
        \StatexIndent [2] $v_l = \bvlshr(\bvshl(\overline{a}, m_l), m_l)$
      \EndCase
      \Case{$v_h\ v_l \leftarrow (a_1.a_2) \ll n$}
        \State{$m_n \leftarrow \bvconst(\wordsize, n)$}
        \State{$r \leftarrow \bvshl(\bvconcat(\overline{a_1}, \overline{a_2}), m_n)$}
        \State{\Return $v_h = \bvhigh(r) \wedge$}
        \StatexIndent [2] $v_l = \bvlshr(\bvlow(r), m_n)$
      \EndCase
    \EndMatch
    \EndFunction
  \end{algorithmic}
  \caption{Transformation from $\bvStmt$ to $\qPred$}
  \label{algorithm:stmt-to-qfbv}
\end{algorithm}

For the transformation from range predicates to QF\_BV formulas, recursively define a function $\textsc{PredrQFBV}$ such that $\textsc{PredrQFBV}($$\top)$ $=$ $\top$, $\textsc{PredrQFBV}($$a_1$ $<$ $a_2)$ $=$ $\bvult($$\overline{a_1},$ $\overline{a_2})$, $\textsc{PredrQFBV}($$a_1$ $\leq$ $a_2)$ $=$ $\bvule($$\overline{a_1},$ $\overline{a_2})$, and $\textsc{PredrQFBV}($$p_1$ $\wedge$ $p_2)$ $=$ $\textsc{PredrQFBV}($$p_1)$ $\wedge$ $\textsc{PredrQFBV}($$p_2)$.
We have the following theorem for the transformation of range predicates.

\begin{theorem}
Let $q \in \bvPredr$ be a range predicate.
Then, for all $\bvst \in \bvSt$, $\BV^\wordsize \models q[\bvst]$ if and only if $\BV^\wordsize \models \textsc{PredrQFBV}(q)[\bvst]$.
\end{theorem}

Define a function $\textsc{StmtSafeQFBV}$ (Algorithm~\ref{algorithm:stmt-safe-qfbv}) which transforms safety checks for statements to QF\_BV.
Recursively define a function $\textsc{ProgSafeQFBV}$ such that $\textsc{ProgSafeQFBV}(\epsilon) \defn \top$ and $\textsc{ProgSafeQFBV}(s; p) \defn \textsc{StmtSafeQFBV}(s) \wedge \textsc{ProgSafeQFBV}(p)$.
The following theorem states the soundness of our translation from range problems and safety checks to QF\_BV.

\begin{theorem}
Given two range predicates $q_r, q_r' \in \bvPredr$ and a well-formed program $p \in \bvProg$ in static single assignment form,
\begin{itemize}
\item $\BV^\wordsize \models q_r[\bvst]$ implies $\textsc{ProgSafe}(p, \bvst) = \btt$ for all $\bvst \in \bvSt$ if, $(\textsc{PredrQFBV}(q_r) \wedge \textsc{ProgQFBV}(p)) \Rightarrow \textsc{ProgSafeQFBV}(p)$ is valid, and
\item $\models \hoaretriple{q_r}{p}{q_r'}$ if the QF\_BV formula $\textsc{PredrQFBV}(q_r)$ $\wedge$ $\textsc{ProgQFBV}(p)$ $\Rightarrow$ $\textsc{PredrQFBV}(q_r')$ is valid.
\end{itemize}
\label{theorem:to-qfbv}
\end{theorem}


\begin{algorithm}
  \begin{algorithmic}[1]
    \Function{StmtSafeQFBV}{$s$}
    \State{$o \leftarrow \bvconst(\wordsize, \bv^\wordsize(0))$}
    \Match{$s$}
      \Case{$v \leftarrow a$} \Return $\top$ \EndCase
      \Case{$v \leftarrow a_1 + a_2$} \Return $\bvhigh(\bvadd^\#(\overline{a_1}, \overline{a_2})) = o$ \EndCase
      \Case{$c\ v \leftarrow a_1 + a_2$} \Return $\top$ \EndCase
      \Case{$v \leftarrow a_1 + a_2 + y$}
        \State{\Return $\bvhigh(\bvadd^\#(\overline{a_1}, \overline{a_2})) = o \wedge$}
        \StatexIndent [2] $\bvhigh(\bvadd^\#(\bvadd^\#(\overline{a_1}, \overline{a_2})), y) = o$
      \EndCase
      \Case{$c\ v \leftarrow a_1 + a_2 + y$} \Return $\top$ \EndCase
      \Case{$v \leftarrow a_1 - a_2$} \Return $\bvhigh(\bvsub^\#(\overline{a_1}, \overline{a_2})) = o$ \EndCase
      \Case{$v \leftarrow a_1 \times a_2$} \Return $\bvhigh(\bvmul^\#(\overline{a_1}, \overline{a_2})) = o$ \EndCase
      \Case{$v_h\ v_l \leftarrow a_1 \times a_2$} \Return $\top$ \EndCase
      \Case{$v \leftarrow a \ll n$}
        \State{$\mathit{one} \leftarrow \bvconst(\wordsize, \bv^\wordsize(1))$}
        \State{$m \leftarrow \bvconst(\wordsize, \bv^\wordsize(\wordsize - |n|))$}
        \State{\Return $\bvult(\overline{a}, \bvshl(\mathit{one}, m))$}
      \EndCase
      \Case{$v_h\ v_l \leftarrow a@n$} \Return $\top$ \EndCase
      \Case{$v_h\ v_l \leftarrow (a_1.a_2) \ll n$}
        \State{$\mathit{one} \leftarrow \bvconst(\wordsize, \bv^\wordsize(1))$}
        \State{$m \leftarrow \bvconst(\wordsize, \bv^\wordsize(\wordsize - |n|))$}
        \State{\Return $\bvult(\overline{a_1}, \bvshl(\mathit{one}, m))$}
      \EndCase
    \EndMatch
    \EndFunction
  \end{algorithmic}
  \caption{Transformation from Safety Checks to QF\_BV}
  \label{algorithm:stmt-safe-qfbv}
\end{algorithm}

