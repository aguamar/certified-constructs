
One of the big issues with modern cryptography is how the assumptions
match up with reality. In many situations, unexpected channels
through which information can leak to the attacker may cause the
cryptosystem to be broken. Typically this is about timing or
electric power used.
In side-channel resilient implementations, the execution time
are kept constant (as much as possible) to prevent unexpected
information leakage.
%Implementations which are not defeated through
%such means are called ``side-channel resilient'' implementations.
Constant execution time however is harder to achieve than one would imagine.
%This turns out in some cases to be harder than one would imagine.
%This is chiefly because
Modern processors have caches
%and out of order execution
and multitasking. This makes it possible for one execution
thread, even when no privilege is conferred, to affect the running
time of another -- simply by caching a sufficient amount of its own
data in correct locations through repeated accesses, and then
observing the running time of the other thread. The instructions in
the other thread which uses the ``evicted'' data (to make room for the
data of the eavesdropping thread) then has to take more time getting
its data back to the cache~\cite{B:05:CTAA}.
% Such attacks are quite practical.
% \todo{insert reference in bib for D.J. Bernstein ``Cache Timing
% Attacks on AES'', https://cr.yp.to/antiforgery/cachetiming-20050414.pdf}


\hide{
As such, modern cryptographic implementations often have to resort to
seemingly contrived sequences of conditional moves and arithmetic
manipulations to achieve the same result as simply using the CPU
instructions designed for the purpose. This of course slows the
computation down and waste resources but is inevitable because
secret-dependent conditional actions and table lookups are actually
very useful tasks that sometimes simply must be performed.
}

Thus, the innocuous actions of executing (a) a conditional
branch instruction dependent on a secret bit, and (b) an
indirect load instruction using a secret value in the register as the
address, are both potentially dangerous leaks of information.
%There are some side results of the above,
Consequently, we are not often
faced with secret-dependent branching or table-lookups in the assembly
instructions, but a language describing cryptographic code might
include pseudo-instructions to cover instruction sequences, phrases in
the language if you will, that is used to achieve the same effect.
The domain specific language \bvdsl is designed based on the same
principles. Conditional branches and indirect memory accesses are not
admitted in \bvdsl.

Assume a fixed wordsize $w$ divisible by $4$.
We use $0xn$ to denote a bit-vector $n$ encoded in hexadecimal format.
A program is a straight line of instructions over bit-vectors with bit-width $w$.
\[
\begin{array}{rcl}
  \Var & ::= & {x} \ |\ {y} \ |\ {z} \ |\ \cdots \\
  \bvAtom & ::= & \Var \ |\ \BV^w \\
  \bvStmt & ::= & \Var \leftarrow \bvAtom \\
          &     & |\ \Var \leftarrow \bvAtom {+} \bvAtom \\
          &     & |\ \Var\ \Var \leftarrow \bvAtom {+} \bvAtom \\
          &     & |\ \Var \leftarrow \bvAtom {+} \bvAtom {+} \Var \\
          &     & |\ \Var\ \Var \leftarrow \bvAtom {+} \bvAtom {+} \Var  \\
          &     & |\ \Var \leftarrow \bvAtom {-} \bvAtom \\
          &     & |\ \Var \leftarrow \bvAtom {*} \bvAtom \\
          &     & |\ \Var\ \Var \leftarrow \bvAtom {*} \bvAtom \\
          &     & |\ \Var \leftarrow \bvAtom {<<} \BV^w \\
          &     & |\ \Var\ \Var \leftarrow \bvAtom {@} \BV^w \\
          &     & |\ \Var\ \Var \leftarrow (\bvAtom {.} \bvAtom) {<<} \BV^w
\end{array}
\]

Let $\bvSt \defn \Var \rightarrow \BV^w$ and $\bvst \in \bvSt$ be a \emph{state} (or \emph{valuation}).
That is, a {state} $\bvst$ is a mapping from variables to bit-vectors in $\BV^w$.
Define
$
\bvst\mymapsto{v}{d}(u) \defn
\left\{
   \begin{array}{ll}
     d & \textmd{if $u = v$}\\
     \bvst(u) & \textmd{otherwise}
   \end{array}
\right.
$.
Define the semantic function $\bvvalueof{\cdot}(\bvst)$ for variables and atomics as follows.
\[
\begin{array}{rcl}
%\valueof{i}_{\bvst} & \defn & i \textmd{  for ${i} \in \Int$} \\
%\valueof{b}_{\bvst} & \defn & b \textmd{  for ${b} \in \BV^w$} \\
\bvvalueof{v}(\bvst) & \defn & \bvst(v) \textmd{  for ${v} \in \Var$} \\
\bvvalueof{a}(\bvst) & \defn & \left\{
  \begin{array}{ll}
  \bvvalueof{v}(\bvst) & \textmd{if $a$ is a variable $v$} \\
  b & \textmd{if $a$ is a bit-vector $b$} \\
  \end{array}
  \right. \\
\end{array}
\]
Consider the transition relation $\bvTr \subseteq \bvSt \times \bvStmt \times \bvSt$ defined in Figure~\ref{fig:semantic-function-bvdsl}.
Basically, $r \leftarrow a_1 + a_2$ is addition, $c\ r \leftarrow a_1 + a_2$ is addition with carry bit placed in $c$, $r \leftarrow a_1 + a_2 + v$ is addition of atomics plus a variable $v$, $c\ r \leftarrow a_1 + a_2 + v$ is addition of atomics plus a variable $v$ with carry bit placed in $c$, $r \leftarrow a_1 - a_2$ is subtraction, $r \leftarrow a_1 * a_2$ is multiplication, $h\ l \leftarrow a_1 * a_2$ is full multiplication, $r \leftarrow a << n$ is left-shifting, $h\ l \leftarrow a @ n$ is splitting at position $n$, and $h\ l \leftarrow (a_1 . a_2) << n$ is left-shifting of $n$ high bits from $a_2$ to $a_1$.
The variable $v$ in $r \leftarrow a_1 + a_2 + v$ and $c\ r \leftarrow a_1 + a_2 + v$ is intended but not restricted to be carry bits.
For $\bvst, \bvst' \in \bvSt$ and $s \in \bvStmt$, we write $\bvst \goesto{s} \bvst'$ when $(\bvst, s, \bvst') \in \bvTr$.

\begin{figure*}
\[
\begin{array}{rcl}
\bvst & \goesto{r \leftarrow a_1 + a_2} & \bvst[r \leftarrow \bvvalueof{a_1}(\bvst) +_\bvop \bvvalueof{a_2}(\bvst)] \\
\bvst & \goesto{c\ r \leftarrow a_1 + a_2} & \bvst[r \leftarrow \bvvalueof{a_1}(\bvst) +_\bvop^\lo \bvvalueof{a_2}(\bvst)][c \leftarrow \bvvalueof{a_1}(\bvst) +_\bvop^\hi \bvvalueof{a_2}(\bvst)] \\
\bvst & \goesto{r \leftarrow a_1 + a_2 + v} & \bvst[r \leftarrow \lo_\bvop(\bvvalueof{a_1}(\bvst) +_\bvop^\# \bvvalueof{a_2}(\bvst) +_\bvop^\# \bvvalueof{v}(\bvst))] \\
\bvst & \goesto{c\ r \leftarrow a_1 + a_2 + v} & \bvst[r \leftarrow \lo_\bvop(\bvvalueof{a_1}(\bvst) +_\bvop^\# \bvvalueof{a_2}(\bvst) +_\bvop^\# \bvvalueof{v}(\bvst))][c \leftarrow \hi_\bvop(\bvvalueof{a_1}(\bvst) +_\bvop^\# \bvvalueof{a_2}(\bvst) +_\bvop^\# \bvvalueof{v}(\bvst))] \\
\bvst & \goesto{r \leftarrow a_1 - a_2} & \bvst[r \leftarrow \bvvalueof{a_1}(\bvst) -_\bvop \bvvalueof{a_2}(\bvst)] \\
\bvst & \goesto{r \leftarrow a_1 * a_2} & \bvst[r \leftarrow \bvvalueof{a_1}(\bvst) *_\bvop \bvvalueof{a_2}(\bvst)] \\
\bvst & \goesto{h\ l \leftarrow a_1 * a_2} & \bvst[h \leftarrow \bvvalueof{a_1}(\bvst) *_\bvop^\hi \bvvalueof{a_2}(\bvst)][l \leftarrow \bvvalueof{a_1}(\bvst) *_\bvop^\lo \bvvalueof{a_2}(\bvst)] \\
\bvst & \goesto{r \leftarrow a << n} & \bvst[r \leftarrow \bvvalueof{a}(\bvst) <<_\bvop \unsigned{n}] \\
\bvst & \goesto{h\ l \leftarrow a @ n} & \bvst[h \leftarrow \bvvalueof{a}(\bvst) >>_\bvop \unsigned{n}][l \leftarrow (\bvvalueof{a}(\bvst) <<_\bvop (w -_{\bbfN} \unsigned{n}))  >>_\bvop (w -_{\bbfN} \unsigned{n})] \\
\bvst & \goesto{h\ l \leftarrow (a_1 . a_2) << n} & \bvst[h \leftarrow \hi_\bvop((\bvvalueof{a_1}(\bvst) ._\bvop \bvvalueof{a_2}(\bvst)) <<_\bvop \unsigned{n})][l \leftarrow (\lo_\bvop((\bvvalueof{a_1}(\bvst) ._\bvop \bvvalueof{a_2}(\bvst)) <<_\bvop \unsigned{n})) >>_\bvop \unsigned{n}] \\
\end{array}
\]
\caption{Transition relation $\bvTr$ for \bvdsl. \label{fig:semantic-function-bvdsl}}
\end{figure*}

A \emph{program} is a sequence of statements.
We denote the empty program by $\epsilon$.
\begin{eqnarray*}
  \bvProg & ::= & \epsilon \ |\ \bvStmt; \bvProg
\end{eqnarray*}
Observe that conditional branches are not allowed in our domain specific language to prevent timing attacks.
The semantics of a program is defined by the relation
$\bvTr^* \subseteq \bvSt \times \bvProg \times \bvSt$ where
$(\bvst, \epsilon, \bvst) \in \bvTr^*$ and
$(\bvst, s; p, \bvst'') \in \bvTr^* \textmd{ if }
    \textmd{there is a } \bvst' \textmd{ with }
    (\bvst, s, \bvst') \in \bvTr$ and
    $(\bvst, p, \bvst'') \in \bvTr^*$.
We write $\bvst \goesto{p} \bvst'$ when $(\bvst, p, \bvst') \in \bvTr^*$.

For specifications, $\top$ denotes the Boolean value $\mathit{tt}$.
We allow two kinds of specifications, namely algebraic specifications evaluated on domain $\bbfZ$ and range specifications evaluated on domain $\BV^w$.
Atomic predicates in an algebraic specification include polynomial equations $e_1 = e_2$ and modular polynomial equations $e_1 \equiv e_2 \mod e_3$ where $e_i \in \bvExpa$ is a polynomial expression for $i \in \{1, 2, 3\}$.
An \emph{algebraic predicate} $q_a \in \bvPreda$ is then a conjunction of atomic algebraic predicates.
\[
\begin{array}{rcl}
  \bvExpa & ::= & \Var \ |\ \bbfZ \ |\ - \bvExpa \ |\ \bvExpa + \bvExpa \\
          &     & |\ \bvExpa - \bvExpa \ |\ \bvExpa * \bvExpa \\
  \bvPreda & ::= & \top \ |\ \bvExpa = \bvExpa \\
           &     & |\ \bvExpa \equiv \bvExpa \mod \bvExpa \\
           &     & |\ \bvPreda \wedge \bvPreda \\
\end{array}
\]
Given a state $\bvst \in \bvSt$ and an expression $e \in \bvExpa$, $\zvalueof{e}(\bvst)$ denotes the value of $e$ on $\bvst$.
\[
\begin{array}{rcl}
  \zvalueof{v} & \defn & \unsigned{\bvst(v)} \textmd{ for $v \in \Var$} \\
  \zvalueof{n} & \defn & n \textmd{ for $n \in \bbfZ$} \\
  \zvalueof{- e} & \defn & -_\bbfZ \zvalueof{e}(\bvst) \\
  \zvalueof{e_1 + e_2} & \defn & \zvalueof{e_1}(\bvst) +_\bbfZ \zvalueof{e_2}(\bvst) \\
  \zvalueof{e_1 - e_2} & \defn & \zvalueof{e_1}(\bvst) -_\bbfZ \zvalueof{e_2}(\bvst) \\
  \zvalueof{e_1 * e_2} & \defn & \zvalueof{e_1}(\bvst) *_\bbfZ \zvalueof{e_2}(\bvst) \\
\end{array}
\]
For an algebraic predicate $q_a \in \bvPreda$, we write $\BV^w \models q_a[\bvst]$ if $q_a$ evaluates to $\btt$ using the evaluation function $\zvalueof{e}(\bvst)$ for every subexpression $e$ in $q$.
%For an algebraic predicate $q_a \in \bvPreda$, we write $\BV^w \models q_a[\bvst]$ if one of the following %holds.
%\begin{itemize}
%  \item $q$ is $\top$.
%  \item $q$ is $e_1 = e_2$ and $\zvalueof{e_1}(\bvst)$ equals $\zvalueof{e_2}(\bvst)$.
%  \item $q$ is $e_1 \equiv e_2 \mod e_3$ and $\zvalueof{e_1}(\bvst) \equiv_\bbfZ \zvalueof{e_2}(\bvst) %\mod_\bbfZ \zvalueof{e_3}(\bvst)$.
%  \item $q$ is $q_1 \wedge q_2$, $\BV^w \models q_1$, and $\BV^w \models q_2$.
%\end{itemize}

We admit comparison between atoms in range specifications as atomic range predicates\footnote{In our implementation, comparison between bit-vector expressions is allowed.}.
A \emph{range predicate} $q_r \in \bvPredr$ is a conjunction of atomic range predicates.
\[
\begin{array}{rcl}
  \bvPredr & ::= & \top \ |\ \bvAtom < \bvAtom \ |\ \bvAtom \leq \bvAtom \\
           &     & |\ \bvPredr \wedge \bvPredr \\
\end{array}
\]
We use $a_l \circ a_1, a_2, \ldots, a_n \bullet a_r$ as a shorthand of the conjunction of $a_l \circ a_1 \wedge a_l \circ a_2 \wedge \cdots \wedge a_l \circ a_n$ and $a_1 \bullet a_r \wedge a_2 \bullet a_r \wedge \cdots \wedge a_n \bullet a_r$ where $\circ, \bullet \in \{<, \leq\}$.
We write $\BV^w \models q_r[\bvst]$ if one of the following holds.
\begin{itemize}
  \item $q$ is $\top$.
  \item $q$ is $a_1 < a_2$ and $\bvvalueof{a_1}(\bvst) <_\bvop \bvvalueof{a_2}(\bvst)$.
  \item $q$ is $a_1 \leq a_2$ and $\bvvalueof{a_1}(\bvst) \leq_\bvop \bvvalueof{a_2}(\bvst)$.
  \item $q$ is $q_1 \wedge q_2$, $\BV^w \models q_1$, and $\BV^w \models q_2$.
\end{itemize}

A \emph{predicate} $q \in \bvPred$ is a conjunction of an algebraic predicate and a range predicate.
\[
\begin{array}{rcl}
  \bvPred & ::= & \bvPreda \wedge \bvPredr \\
  \bvSpec & ::= & \cond{\bvPred} \bvProg \cond{\bvPred}
\end{array}
\]
For $\bvst \in \bvSt$ and $q \in \bvPred$, we write $\BV^w \models q[\bvst]$ if $q$ evaluates to $\btt$; $\bvst$ is called a \emph{$q$-state}.
We follow Hoare's formalism in specifications of mathematical constructs~\cite{H:69:ABCP}.
In a specification $\hoaretriple{q}{p}{q'}$, $q$ and $q'$ are the \emph{pre}- and \emph{post-conditions} of $p$ respectively.
Given $q, q' \in \bvPred$ and $p \in \bvProg$, $\hoaretriple{q}{p}{q'}$ is \emph{valid} (written $\models \hoaretriple{q}{p}{q'}$) if for every $\bvst, \bvst' \in \bvSt$, $\BV^w \models q[\bvst]$ and $\bvst \goesto{p} \bvst'$ imply $\BV^w \models q'[\bvst']$.
Less formally, $\models \hoaretriple{q}{p}{q'}$ if executing $p$ from a $q$-state always results in a $q'$-state.

\begin{figure*}[ht]
  \centering
  \[
  \begin{array}{lclcl}
    \begin{array}{rcl}
    \textmd{1:} && {r}_0 \leftarrow {x}_0; \\
    \textmd{2:} && {r}_1 \leftarrow {x}_1; \\
    \textmd{3:} && {r}_2 \leftarrow {x}_2; \\
    \textmd{4:} && {r}_3 \leftarrow {x}_3; \\
    \textmd{5:} && {r}_5 \leftarrow {x}_4; \\
    \end{array}
    &\hspace{.05\textwidth}&
    \begin{array}{rcl}
    \textmd{6:} &&
      {r}_0 \leftarrow {r}_0 + {0xFFFFFFFFFFFDA}; \\
    \textmd{7:} &&
      {r}_1 \leftarrow {r}_1 + {0xFFFFFFFFFFFFE}; \\
    \textmd{8:} &&
      {r}_2 \leftarrow {r}_2 + {0xFFFFFFFFFFFFE}; \\
    \textmd{9:} &&
      {r}_3 \leftarrow {r}_3 + {0xFFFFFFFFFFFFE}; \\
    \textmd{10:} &&
      {r}_4 \leftarrow {r}_4 + {0xFFFFFFFFFFFFE};\\
    \end{array}
    &\hspace{.05\textwidth}&
    \begin{array}{rcl}
    \textmd{11:} && {r}_0 \leftarrow {r}_0 - {y}_0; \\
    \textmd{12:} && {r}_1 \leftarrow {r}_1 - {y}_1; \\
    \textmd{13:} && {r}_2 \leftarrow {r}_2 - {y}_2; \\
    \textmd{14:} && {r}_3 \leftarrow {r}_3 - {y}_3; \\
    \textmd{15:} && {r}_4 \leftarrow {r}_4 - {y}_4;
    \end{array}
  \end{array}
  \]
  \caption{Subtraction \dslcode{sub}}
  \label{figure:dsl:subtraction}
\end{figure*}

Figure~\ref{figure:dsl:subtraction} gives a simple yet real implementation of subtraction over $\bbfGF(\myprime)$ with a bit-width $64$.
In the figure, a number in $\bbfGF(\varrho)$ is represented by five bit-vectors each with value less than or equal to $2^{51} +_\bbfZ 2^{15}$.
The variables ${x}_0, {x}_1, {x}_2, {x}_3, {x}_4$ for instance represent $\mathit{radix51}({x}_4, {x}_3, {x}_2, {x}_1, {x}_0) \defn 2^{51 \times 4} {x}_4 + 2^{51 \times 3} {x}_3 + 2^{51 \times 2} {x}_2 + 2^{51 \times 1} {x}_1 + 2^{51 \times 0} {x}_0$.
The result of subtraction is stored in the variables ${r}_0, {r}_1, {r}_2, {r}_3, {r}_4$, which are all required to be in the range from $0$ to $2^{54}$.
Let $q_a$ $\defn$ $\top$, $q_r$ $\defn$ $0$ $\leq$ ${x}_0,$ ${x}_1,$ ${x}_2,$ ${x}_3,$ ${x}_4,$ ${y}_0,$ ${y}_1,$ ${y}_2,$ ${y}_3,$ ${y}_4$ $\leq$ $\bv^{64}(2^{51} +_\bbfZ 2^{15})$, $q_a'$ $\defn$ $\mathit{radix51}({x}_4, {x}_3, {x}_2, {x}_1, {x}_0)$ $-$ $\mathit{radix51}({y}_4, {y}_3, {y}_2, {y}_1, {y}_0)$ $\equiv$ $\mathit{radix51}({r}_4, {r}_3, {r}_2, {r}_1, {r}_0)$ $\mod$ $\myprime$, and $q_r'$ $\defn$ $0$ $\leq$ ${r}_0,$ ${r}_1,$ ${r}_2,$ ${r}_3,$ ${r}_4$ $<$ $\bv^{64}(2^{54})$.
The specification of the mathematical construct is therefore
\[
\begin{array}{rcl}
\cond{q_a \wedge q_r} &
\dslcode{sub} &
\cond{q_a' \wedge q_r'}.
\end{array}
\]

% 4503599627370458 = 0xFFFFFFFFFFFDA
% 4503599627370494 = 0xFFFFFFFFFFFFE

Note that the variables ${r}_i$'s are added with constants after they are initialized with ${x}_i$'s but before ${y}_i$'s are subtracted from them.
Assume the program \dslcode{sub} does not overflow or underflow.
It is not hard to see that $2\myprime = \mathit{radix51}$ $(\unsigned{0xFFFFFFFFFFFFE},$ $\unsigned{0xFFFFFFFFFFFFE},$ $\unsigned{0xFFFFFFFFFFFFE},$ $\unsigned{0xFFFFFFFFFFFFE},$ $\unsigned{0xFFFFFFFFFFFDA})$ after tedious computation.
Hence $\mathit{radix51}({r}_4,$ ${r}_3,$ ${r}_2,$ ${r}_1,$ ${r}_0)$ $=$ $\mathit{radix51}({x}_4,$ ${x}_3,$ ${x}_2,$ ${x}_1,$ ${x}_0)$ $+$ $2 \myprime $ $-$
$\mathit{radix51}({y}_4,$ ${y}_3,$ ${y}_2,$ ${y}_1,$ ${y}_0)$ $\equiv$ $\mathit{radix51}({x}_4,$ ${x}_3,$ ${x}_2,$ ${x}_1,$ ${x}_0)$ $-$ $\mathit{radix51}({y}_4,$ ${y}_3,$ ${y}_2,$ ${y}_1,$ ${y}_0)$ $\mod\ \myprime$.
The program in Figure~\ref{figure:dsl:subtraction} is correct.
Characteristics of large Galois fields are regularly exploited in mathematical constructs for correctness and efficiency.
Our domain specific language can easily model such specialized programming techniques.
Indeed, the reason for adding constants is to prevent underflow.
If the constants were not added, the subtraction in lines~11 to 15 could give negative and hence incorrect results.
We will show how to prove that the program does not overflow or underflow later.

%{Please note that ranges are a complicated
%  issue. The subtraction in Figure~\ref{figure:dsl:subtraction} gives results that are correct
%  but possibly {overflowing} ($>\varrho$), which can and must be
%  accounted for later.}
