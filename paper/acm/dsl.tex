Algebraic specifications in \bvdsl are transformed to modular polynomial equation entailment problems via an intermediate language \zdsl.
A program in \zdsl is but a straight line of variable assignments on expressions.
Consider the following syntactic classes:
\[
\begin{array}{rcl}
  \zExpr & ::= & \bbfZ \ |\ \Var \ |\  {-}\zExpr \ |\
                 \zExpr {+} \zExpr \ |\ \zExpr {-} \zExpr \\
  & & |\ \zExpr {\times} \zExpr  \ |\ \dslcode{Pow} (\zExpr, \bbfN)
\end{array}
\]

We allow exact integers as constants in \zdsl.
Variables are thus integer variables. An expression can be a
constant, a variable, or a negative expression. Additions, subtractions,
and multiplications of expressions are available.
The expression $\dslcode{Pow}(e, n)$ denotes $e^n$ for any expression $e$ and natural number $n$.
More formally, let $\zSt \defn \Var \rightarrow
\bbfZ$ and $\zst \in \zSt$ be a state. That is,
a {state} $\zst$ in \zdsl is a mapping from variables to integers. Define the
semantic function $\zvalueof{e}(\zst)$ as follows.
\[
\begin{array}{rcl}
  \zvalueof{{i}}(\zst) & \defn & i \textmd{  for ${i} \in \bbfZ$} \\
  \zvalueof{{v}}(\zst) & \defn & \zst({v}) \textmd{  for ${v} \in \Var$} \\
  \zvalueof{-e}(\zst) & \defn & -_{\bbfZ}\zvalueof{e}(\zst) \\
  \zvalueof{e_0 + e_1}(\zst) & \defn & \zvalueof{e_0}(\zst) +_{\bbfZ} \zvalueof{e_1}(\zst) \\
  \zvalueof{e_0 - e_1}(\zst) & \defn & \zvalueof{e_0}(\zst) -_{\bbfZ} \zvalueof{e_1}(\zst) \\
  \zvalueof{e_0 \times e_1}(\zst) & \defn & \zvalueof{e_0}(\zst) \times_{\bbfZ} \zvalueof{e_1}(\zst) \\
  \zvalueof{\dslcode{Pow}(e, n)}(\zst) & \defn & (\zvalueof{e}(\zst))^n
\end{array}
\]

\hide{
Note that there is no expression for quotients, or bitwise logical
operations in \zdsl. Bitwise left shifting however can be modeled by
multiplying $\dslcode{Pow}(2, n)$. Although \zdsl models a (very) 
small subset of assembly, it suffices to encode low-level
mathematical constructs in X25519.
}

In \zdsl, only assignments are allowed.
The statement $v \leftarrow e$ assigns the value of $e$
to the variable $v$. For bounded additions, multiplications, and right
shifting, they are modeled by the construct $\dslcode{Split}$ in \zdsl.
The statement $\concat{v_h}{v_l} \leftarrow
\dslcode{Split}(e, n)$ splits the value of $e$ into two parts;
the lowest $n$ bits are stored in $v_l$ and the remaining higher bits
are stored in $v_h$.
Consider the relation $\zTr \subseteq \zSt \times \zStmt \times \zSt$ defined
by $(\zst, v \leftarrow e, \zst\mymapsto{v}{\zvalueof{e}(\zst)}) \in \zTr$, and
  $(\zst, \concat{v_h}{v_l} \leftarrow \dslcode{Split} (e, n),
  \zst\mymapsto{v_h}{\mathit{hi}}\mymapsto{v_l}{\mathit{lo}}) \in \zTr$
where
$\mathit{hi} = (\zvalueof{e}(\zst) - lo) \div 2^{n}$ and
$\mathit{lo} = \zvalueof{e}(\zst) \mod 2^{n}$.
Intuitively, $(\zst, s, \zst') \in \zTr$ denotes that the state $\zst$ transits to
the state $\zst'$ after executing the statement $s$.
\begin{eqnarray*}
  \zStmt & ::= & \Var \leftarrow \zExpr
            \ |\  \concat{\Var}{\Var} \leftarrow \dslcode{Split} (\zExpr, \bbfN) \\
  \zProg & ::= & \epsilon \ |\ \zStmt; \zProg
\end{eqnarray*}
A program is a sequence of statements. Again, we denote the empty program by
$\epsilon$.
\hide{
Observe that conditional branches are not allowed in our
domain specific language to prevent timing attacks.
}
The semantics of a program is defined by the relation
$\zTr^* \subseteq \zSt \times \zProg \times \zSt$ where
$(\zst, \epsilon, \zst) \in \zTr^*$ and
$(\zst, s; p, \zst'') \in \zTr^* \textmd{ if }
    \textmd{there is a } \zst' \textmd{ with }
    (\zst, s, \zst') \in \zTr$ and
    $(\zst, p, \zst'') \in \zTr^*$.
We write $\zst \goesto{p} \zst'$ when $(\zst, p, \zst') \in \zTr^*$.

\hide{
For algebraic specifications in \zdsl, $\top$ denotes the Boolean value
$\mathit{tt}$. We admit polynomial equations $e = e'$ and modular
polynomial equations $e \equiv e' \mod n$ as atomic predicates.
A \emph{predicate} $q \in \zPred$ is a conjunction of atomic
predicates. For
$\zst \in \zSt$, write $\bbfZ \models q[\zst]$ if $q$
evaluates to $\btt$ after replacing each variable $v$ with
$\zst(v)$; $\zst$ is called a \emph{$q$-state}.
\begin{eqnarray*}
  \Pred & ::= & \top
     \ |\   \Expr = \Expr
     \ |\   \Expr \equiv \Expr \mod \Nat
     \ |\   \Pred \wedge \Pred\\
  \Spec & ::= & \cond{\Pred} \Prog \cond{\Pred}
\end{eqnarray*}
We follow Hoare's formalism in algebraic specifications of
mathematical constructs~\cite{H:69:ABCP}. In an algebraic
specification $\hoaretriple{q}{p}{q'}$, $q$ and $q'$ are the \emph{pre}- and
\emph{post-conditions} of $p$ respectively. Given $q, q' \in \Pred$ and $p
\in \Prog$, $\hoaretriple{q}{p}{q'}$ is \emph{valid}
(written $\models \hoaretriple{q}{p}{q'}$) if for every $\zst, \zst' \in
\St$, $\bbfZ \models q[\zst]$ and $\zst \goesto{p} \zst'$ imply
$\bbfZ \models q'[\zst']$. Less formally, $\models
\hoaretriple{q}{p}{q'}$ if executing $p$ from a $q$-state always
results in a $q'$-state.
}

The predicates $\zPred$ in \zdsl share the same syntax as the algebraic predicates in \bvdsl but are evaluated on $\zSt$ rather than on $\bvSt$.
\[
\begin{array}{rcl}
  \zPred & ::= & \top \ |\ \zExpr = \zExpr\ |\ \zExpr \equiv \zExpr \mod \zExpr \\
         &     & |\ \zPred \wedge \zPred \\
\end{array}
\]
For $\zst \in \zSt$ and $q \in \bvPreda$, write $\bbfZ \models q[\zst]$ if $q$ evaluates to $\btt$ using the evaluation function $\zvalueof{e}(\zst)$ for every subexpression $e$ in $q$.
%after replacing each variable $v$ with $\zst(v)$.
Given $q, q' \in \zPred$ and $p \in \zProg$, $\hoaretriple{q}{p}{q'}$ is valid (written $\models \hoaretriple{q}{p}{q'}$) if for every $\zst, \zst' \in \zSt$, $\bbfZ \models q[\zst]$ and $\zst \goesto{p} \zst'$ imply $\bbfZ \models q'[\zst']$.

\hide{
\begin{figure*}[ht]
  \centering
  \[
  \begin{array}{lclcl}
    \begin{array}{rcl}
    \textmd{1:} && {r}_0 \leftarrow {x}_0; \\
    \textmd{2:} && {r}_1 \leftarrow {x}_1; \\
    \textmd{3:} && {r}_2 \leftarrow {x}_2; \\
    \textmd{4:} && {r}_3 \leftarrow {x}_3; \\
    \textmd{5:} && {r}_5 \leftarrow {x}_4; \\
    \end{array}
    &\hspace{.05\textwidth}&
    \begin{array}{rcl}
    \textmd{6:} &&
      {r}_0 \leftarrow {r}_0 + {4503599627370458}; \\
    \textmd{7:} &&
      {r}_1 \leftarrow {r}_1 + {4503599627370494}; \\
    \textmd{8:} &&
      {r}_2 \leftarrow {r}_2 + {4503599627370494}; \\
    \textmd{9:} &&
      {r}_3 \leftarrow {r}_3 + {4503599627370494}; \\
    \textmd{10:} &&
      {r}_4 \leftarrow {r}_4 + {4503599627370494};\\
    \end{array}
    &\hspace{.05\textwidth}&
    \begin{array}{rcl}
    \textmd{11:} && {r}_0 \leftarrow {r}_0 - {y}_0; \\
    \textmd{12:} && {r}_1 \leftarrow {r}_1 - {y}_1; \\
    \textmd{13:} && {r}_2 \leftarrow {r}_2 - {y}_2; \\
    \textmd{14:} && {r}_3 \leftarrow {r}_3 - {y}_3; \\
    \textmd{15:} && {r}_4 \leftarrow {r}_4 - {y}_4;
    \end{array}
  \end{array}
  \]
  \caption{Subtraction \dslcode{sub} in \zdsl}
%$(\dslcode{x}_0, \dslcode{x}_1, \dslcode{x}_2, \dslcode{x}_3,
%\dslcode{x}_4, \dslcode{y}_0, \dslcode{y}_1, \dslcode{y}_2,
%\dslcode{y}_3, \dslcode{y}_4)$
  \label{figure:dsl:subtraction}
\end{figure*}
}

Now we are ready to describe the transformation from an algebraic specification in \bvdsl to a specification in \zdsl.
Given $\bvst \in \bvSt$ and $\zst \in \zSt$, write $\bvst \simeq \zst$ when $|\bvst(v)| = \zst(v)$ for all variable $v \in \Var$.
For algebraic expressions, since $\zExpr$ subsumes $\bvExpa$, we can easily define a function \textsc{BV2ZExpr} that converts an algebraic expression $e_a \in \bvExpa$ to $\zExpr$ such that for every $\bvst \in \bvSt$ and $\zst \in \zSt$ with $\bvst \simeq \zst$, $\zvalueof{e_a}(\bvst) = \zvalueof{\textsc{BV2ZExpr}(e_a)}(\zst)$.
Similarly, we can define a function \textsc{BV2ZPred} such that for every $q_a \in \bvPreda$, $\bvst \in \bvSt$, and $\zst \in \zSt$ with $\bvst \simeq \zst$, $\BV^{\wordsize} \models q_a[\bvst]$ if and only if $\bbfZ \models \textsc{BV2ZPred}(q_a)[\zst]$.
Atoms are translated by the function $\textsc{BV2ZAtom}$.
\[
\textsc{BV2ZAtom}(a) = \left\{
  \begin{array}{ll}
  |b| & \textmd{ if $a$ is a bit-vector $b$} \\
  v   & \textmd{ if $a$ is a variable $v$}
  \end{array} \right.
\]
Let $\tilde{a}$ denote $\textsc{BV2ZAtom}(a)$ for $a \in bvAtom$.
Functions \textsc{BV2ZStmt} (Algorithm~\ref{algorithm:bv2z-stmt}) and \textsc{BV2ZProg} (Algorithm~\ref{algorithm:bv2z-prog}) are defined to respectively transform a statement in \bvdsl to a statement in \zdsl and a program in \bvdsl to a program in \zdsl.
With these translation functions, the following soundness theorem holds.
\begin{theorem}
  For every $q_a, q_a' \in \bvPreda$, $q_r, q_r' \in \bvPredr$, and $p \in \bvProg$, $\models \hoaretriple{q_a \andar q_r}{p}{q_a' \andar q_r'}$ if all the following conditions hold.
  \begin{itemize}
  \item[C1] $\BV^\wordsize \models q_r[\bvst]$ implies $\textsc{ProgSafe}(p, \bvst) = \btt$ for all $\bvst \in \bvSt$.
  \item[C2] $\models \hoaretriple{q_r}{p}{q_r'}$.
  \item[C3] $\models \hoaretriple{\textsc{BV2ZPred}(q_a)}{\textsc{BV2ZProg}(p)}{\textsc{BV2ZPred}(q_a')}$.
  \end{itemize}
  \label{theorem:bv2z}
\end{theorem}
As conditions C1 and C2 involve only bit-vector operations, both conditions can be verified by translations to the QF\_BV fragment (quantifier-free formulas over the theory of fixed-size bit-vectors) of SMT (Section~\ref{subsection:solving-range-overflow-checks}).
Condition C3 is verified by a transformation to polynomial equation entailment (Section~\ref{subsection:solving-algebraic-equations}).
Note that the inverse implication of Theorem~\ref{theorem:bv2z} does not hold because for example, $\models \hoaretriple{q_r}{p}{q_r'}$ may require that $q_a$ holds initally but we do not consider any algebraic predicates in verifying range specifications.

\hide{
\begin{algorithm}
  \begin{algorithmic}[1]
    \Function{BV2ZAtom}{$a$}
    \Match{$a$}
      \Case{$b$} \Return{$|b|$} \EndCase
      \Case{$v$} \Return{v} \EndCase
    \EndMatch
    \EndFunction
  \end{algorithmic}
  \caption{Transformation from $\bvAtom$ to $\zExpr$}
  \label{algorithm:bv2z-atom}
\end{algorithm}
}

\begin{algorithm}
  \begin{algorithmic}[1]
    \Function{BV2ZStmt}{$s$}
    \Match{$s$}
      \Case{$v \leftarrow a$}
        \Return $v \leftarrow \tilde{a}$
      \EndCase
      \Case{$v \leftarrow a_1 + a_2$}
        \Return $v \leftarrow \tilde{a_1} + \tilde{a_2}$
      \EndCase
      \Case{$c\ v \leftarrow a_1 + a_2$}
        \State{\Return $\concat{c}{v} \leftarrow \dslcode{Split}(\tilde{a_1} + \tilde{a_2}, w)$}
      \EndCase
      \Case{$v \leftarrow a_1 + a_2 + y$}
        \Return $v \leftarrow \tilde{a_1} + \tilde{a_2} + y$
      \EndCase
      \Case{$c\ v \leftarrow a_1 + a_2 + y$}
        \State{\Return $\concat{c}{v} \leftarrow \dslcode{Split}(\tilde{a_1} + \tilde{a_2} + y, w)$}
      \EndCase
      \Case{$v \leftarrow a_1 - a_2$}
        \Return $v \leftarrow \tilde{a_1} - \tilde{a_2}$
      \EndCase
      \Case{$v \leftarrow a_1 \times a_2$}
        \Return $v \leftarrow \tilde{a_1} \times \tilde{a_2}$
      \EndCase
      \Case{$v_h\ v_l \leftarrow a_1 \times a_2$}
        \State{\Return $\concat{v_h}{v_l} \leftarrow \dslcode{Split}(\tilde{a_1} \times \tilde{a_2}, w)$}
      \EndCase
      \Case{$v \leftarrow a \ll n$}
        \Return $v \leftarrow \tilde{a} \times \dslcode{Pow}(2, |n|)$
      \EndCase
      \Case{$v_h\ v_l \leftarrow a@n$}
        \Return $\concat{v_h}{v_l} \leftarrow \dslcode{Split}(\tilde{a}, |n|)$
      \EndCase
      \Case{$v_h\ v_l \leftarrow (a_1.a_2) \ll n$}
        \State{\Return $\concat{v_h}{v_l} \leftarrow \dslcode{Split}(\tilde{a_1} \times \dslcode{Pow}(2, w) + \tilde{a_2}, w - |n|)$}
      \EndCase
    \EndMatch
    \EndFunction
  \end{algorithmic}
  \caption{Transformation from $\bvStmt$ to $\zStmt$ ($w$ is the assumed wordsize)}
  \label{algorithm:bv2z-stmt}
\end{algorithm}

\begin{algorithm}
  \begin{algorithmic}[1]
    \Function{BV2ZProg}{$p$}
    \Match{$p$}
      \Case{$\epsilon$} \Return{$\epsilon$} \EndCase
      \Case{$s; pp$} \Return{\Call{BV2ZStmt}{$s$}; \Call{BV2ZProg}{$pp$}} \EndCase
    \EndMatch
    \EndFunction
  \end{algorithmic}
  \caption{Transformation from $\bvProg$ to $\zProg$}
  \label{algorithm:bv2z-prog}
\end{algorithm}

The function \textsc{BV2ZProg} preserves well-formedness and static single assignment form. This is showed by the following lemma.

\begin{lemma}
  \label{lemma:bv2z-prog}
  Given a well-formed program $p \in \bvProg$ in static single assignment form, $\textsc{BV2ZProg}(p) \in \zProg$ is well-formed and in static single assignment form.
\end{lemma}

Figure~\ref{figure:translation:subtraction-zdsl} shows the result of transforming the subtraction program \dslcode{bSubSSA} to \zdsl.

\begin{figure*}
  \centering
  $
  \begin{array}{lclcl}
    \begin{array}{rcl}
    \textmd{1:} && {r}^1_0 \leftarrow {x}^0_0; \\
    \textmd{2:} && {r}^1_1 \leftarrow {x}^0_1; \\
    \textmd{3:} && {r}^1_2 \leftarrow {x}^0_2; \\
    \textmd{4:} && {r}^1_3 \leftarrow {x}^0_3; \\
    \textmd{5:} && {r}^1_4 \leftarrow {x}^0_4; \\
    \end{array}
    &\hspace{.05\textwidth}&
    \begin{array}{rcl}
    \textmd{6:} &&
      {r}^2_0 \leftarrow {r}^1_0 + {4503599627370458}; \\
    \textmd{7:} &&
      {r}^2_1 \leftarrow {r}^1_1 + {4503599627370494}; \\
    \textmd{8:} &&
      {r}^2_2 \leftarrow {r}^1_2 + {4503599627370494}; \\
    \textmd{9:} &&
      {r}^2_3 \leftarrow {r}^1_3 + {4503599627370494}; \\
    \textmd{10:} &&
      {r}^2_4 \leftarrow {r}^1_4 + {4503599627370494};\\
    \end{array}
    &\hspace{.05\textwidth}&
    \begin{array}{rcl}
    \textmd{11:} && {r}^3_0 \leftarrow {r}^2_0 - {y}^0_0; \\
    \textmd{12:} && {r}^3_1 \leftarrow {r}^2_1 - {y}^0_1; \\
    \textmd{13:} && {r}^3_2 \leftarrow {r}^2_2 - {y}^0_2; \\
    \textmd{14:} && {r}^3_3 \leftarrow {r}^2_3 - {y}^0_3; \\
    \textmd{15:} && {r}^3_4 \leftarrow {r}^2_4 - {y}^0_4;
    \end{array}
  \end{array}
  $
  \caption{Subtraction \textsc{BV2ZProg}(\dslcode{bSubSSA})}
  \label{figure:translation:subtraction-zdsl}
\end{figure*}


\hide{
Figure~\ref{figure:dsl:subtraction} gives a simple yet real
implementation of subtraction over $\bbfGF(\myprime)$.
In the figure, a number in $\bbfGF(\varrho)$
is represented by five 51-bit non-negative integers. The variables
${x}_0, {x}_1, {x}_2, {x}_3, {x}_4$ for instance represent
$\mathit{radix51}({x}_4, {x}_3, {x}_2, {x}_1, {x}_0) \defn
2^{51 \times 4} {x}_4 + 2^{51 \times 3} {x}_3 +
2^{51 \times 2} {x}_2 + 2^{51 \times 1} {x}_1 +
2^{51 \times 0} {x}_0$. The result of
subtraction is stored in the variables ${r}_0, {r}_1, {r}_2, {r}_3, {r}_4$.
Given $0 \leq {x}_0,$ ${x}_1,$ ${x}_2,$ ${x}_3,$ ${x}_4,$ ${y}_0,$
${y}_1,$ ${y}_2,$ ${y}_3,$ ${y}_4 < 2^{51}$,
the algebraic specification of the mathematical construct is therefore
\[
\begin{array}{rcl}
\cond{\top} &
\dslcode{sub} &
\cond{
   \begin{array}{c}
     \mathit{radix51}({x}_4, {x}_3, {x}_2, {x}_1, {x}_0) -
     \mathit{radix51}({y}_4, {y}_3, {y}_2, {y}_1, {y}_0)\\
     \equiv
     \mathit{radix51}({r}_4, {r}_3, {r}_2, {r}_1, {r}_0)
     \mod \myprime
   \end{array}
}.
\end{array}
\]

Note that the variables ${r}_i$'s are added with constants
after they are initialized with ${x}_i$'s but before
${y}_i$'s are subtracted from them. It is not hard to see that
$2\myprime = \mathit{radix51}$ $(4503599627370494,$ $4503599627370494,$
$4503599627370494,$ $4503599627370494,$ $4503599627370458)$
after tedious computation. Hence $\mathit{radix51}({r}_4,$
${r}_3,$ ${r}_2,$ ${r}_1,$ ${r}_0)$ $=$
$\mathit{radix51}({x}_4,$ ${x}_3,$ ${x}_2,$ ${x}_1,$ ${x}_0)$ $+$ 
$2 \myprime $ $-$
$\mathit{radix51}({y}_4,$ ${y}_3,$ ${y}_2,$ ${y}_1,$ ${y}_0)$ 
$\equiv $
$\mathit{radix51}({x}_4,$ ${x}_3,$ ${x}_2,$ ${x}_1,$ ${x}_0)$ $-$
$\mathit{radix51}({y}_4,$ ${y}_3,$ ${y}_2,$ ${y}_1,$ ${y}_0)$ $\mod 
\ \myprime$. The program in
Figure~\ref{figure:dsl:subtraction} is correct. 
Characteristics of large Galois fields are
regularly exploited in mathematical constructs for correctness and
efficiency. Our domain specific language can easily model such
specialized programming techniques. Indeed,
the reason for
adding constants is to prevent underflow. If the constants were not
added, the subtraction in lines~11 to 15 could give negative and hence
%
incorrect results. {Please note that ranges are a complicated
  issue. The subtraction in Figure~\ref{figure:dsl:subtraction} gives results that are correct
  but possibly {overflowing} ($>\varrho$), which can and must be
  accounted for later.} 
}

%%% Local Variables: 
%%% mode: latex
%%% eval: (TeX-PDF-mode 1)
%%% eval: (TeX-source-correlate-mode 1)
%%% TeX-master: "certified_vcg"
%%% End: 
