
We write $\mathbb{B} = \{ \mathit{ff}, \mathit{tt} \}$ for the Boolean
domain. Let $\bbfN$ and $\bbfZ$ denote all natural numbers and all integers
respectively. We use $[n]$ to denote the set $\{ 0, 1, \ldots, n \}$
for $n \in \bbfN$.

A \emph{monoid} $\mathcal{M} = (M, \epsilon, \cdot)$ consists of a set
$M$ and an associative binary operator $\cdot$ on $M$ with the
\emph{identity} $\epsilon \in M$. That is, $\epsilon \cdot m = m \cdot
\epsilon = m$ for every $m \in M$.
A \emph{group} $\mathcal{G} = (G, 0, +)$ is an algebraic structure
where $(G, 0, +)$ is a monoid and there is a $-a \in G$ such that
$(-a) + a = a + (-a) = 0$ for every $a \in G$. The element $-a$ is
called the \emph{inverse} of $a$. $\mathcal{G}$ is \emph{Abelian} if
the operator $+$ is commutative.
A \emph{ring} $\mathcal{R} = (R, 0, 1, +, \times)$ with $0 \neq 1$ is
an algebraic structure such that
\begin{itemize}
\item $(R, 0, +)$ is an Abelian group; 
\item $(R, 1, \times)$ is a monoid; and 
\item $\times$ is distributive over $+$: $a \times (b + c) = a \times
  b + a \times c$ for every $a, b, c \in R$.
\end{itemize}
If $\times$ is commutative, $\mathcal{R}$ is a \emph{commutative}
ring. 
% An \emph{integral domain} is a commutative ring where $a \times
% b = 0$ implies $a = 0$ or $b = 0$. 
A \emph{field} $\mathcal{F} = (F,
0, 1, +, \times)$ is a commutative ring where $(F\!\setminus\!\{0\}, 1, \times)$ is also
a group. $(\bbfN, 1, \times)$ is a monoid. $(\bbfZ, 0, 1, +, \times)$ 
is a commutative ring but not a field. 
For any prime number $\varrho$, the set $\{ 0, \ldots, \varrho \}$
with the addition and multiplication modulo $\varrho$ forms a \emph{Galois
field} of order $\varrho$ (written $\bbfGF(\varrho)$).
We focus on Galois fields of very large orders, in particular, $\myprime =
2^{255} - 19$.

Fix a set of variables $\vx$. $\mathcal{R}[\vx]$ is the set of
polynomials over $\vx$ with coefficients in the ring
$\mathcal{R}$. $\mathcal{R}[\vx]$ is a ring. A set $I \subseteq
\mathcal{R}[\vx]$ is an \emph{ideal} if 
\begin{itemize}
\item $f + g \in I$ for every $f, g \in I$; and
\item $h \times f \in I$ for every $h \in
  \mathcal{R}[\vx]$ and $f \in I$. 
\end{itemize}
Given $G \subseteq \mathcal{R}[\vx]$, $\langle G \rangle$ is the
minimal ideal containing $G$; $G$ are the \emph{generators}
of $\langle G \rangle$. The \emph{ideal membership}
problem is to decide if $f \in I$ for a given ideal $I$ and $f
\in \mathcal{R}[\vx]$.

Let $\BV^w$ be the set of all bit-vectors with a bit-width $w$.
The unsigned value of $b \in \BV^w$ is denoted by $\unsigned{b}$.
For a natural number or an integer $n$, let $\bv^w(n)$ be the two's complementation representation of $n$ in a bit-width $w$.
Assume the following common operators for fixed-width bit-vectors: $\BV^w +_\bvop \BV^w : \BV^w$ for addition, $\BV^w -_\bvop \BV^w : \BV^w$ for subtraction, $\BV^w *_\bvop \BV^w : \BV^w$ for multiplication, $\BV^{w_1} ._\bvop \BV^{w_2} : \BV^{w_1 + w_2}$ for concatenation, $\BV^w \#_\bvop n : \BV^{w+n}$ for zero extension, $\BV^w <<_\bvop n : \BV^w$ for left-shifting, $\BV^w >>_\bvop n : \BV^w$ for logical right-shifting, $\hi_\bvop(\BV^{2w}) : \BV^w$ for extraction of high bits, and $\lo_\bvop(\BV^{2w}) : \BV^w$ for extraction of low bits.
Let $b_1$, $b_2$, and $b_3$ are bit-vectors in $\BV^{w}$, and $\cdot \in \{+_\bvop, -_\bvop, *_\bvop\}$.
Define three more operations which perform addition, subtraction, or multiplication after zero extension.
\[
\begin{array}{rcl}
b_1 \cdot^{\#} b_2 & \defn & (b_1 \#_\bvop w) \cdot (b_2 \#_\bvop w) \\
\end{array}
\]
When $\cdot^{\#}$ is applied to more than two bit-vectors, bit-vectors with bit-width already in $2w$ will not be extended again.
For example, $b_1 +_\bvop^{\#} b_2 +_\bvop^{\#} b_3 = (b_1 \#_\bvop w) +_\bvop (b_2 \#_\bvop w) +_\bvop (b_3 \#_\bvop w)$.
Define six more operations which takes high bits or low bits after $\cdot^{\#}$ operation.
\[
\begin{array}{rcl}
b_1 \cdot^{\hi} b_2 & \defn & \hi_\bvop(b_1 \cdot^{\#} b_2) \\
b_1 \cdot^{\lo} b_2 & \defn & \lo_\bvop(b_1 \cdot^{\#} b_2) \\
\end{array}
\]
We also assume comparison operators $<_\bvop$ and $\leq_\bvop$ between unsigned values of bit-vectors.


%%% Local Variables:
%%% mode: latex
%%% eval: (TeX-PDF-mode 1)
%%% eval: (TeX-source-correlate-mode 1)
%%% TeX-master: "certified_vcg"
%%% End:
