
In order to take advantages of computer security offered by modern 
cryptography, 
cryptosystems must be realized by cryptographic programs where mathematical
constructs are required to compute on the underlying algebraic
structures of cryptosystems.
Such mathematical constructs are frequently invoked in cryptographic
programs. They are often written in low-level assembly languages and
manually optimized for efficiency. 
% Cryptographic primitives in cryptosystems are but sequences of
% algebraic operations on mathematical structures. 
Security of cryptosystems could be compromised should programming
errors in mathematical constructs be exploited by adversaries.
Security guarantees of cryptographic programs
thus depend heavily on the correctness of mathematical constructs.
%It is therefore of utmost importance to ensure the correctness of
%mathematical constructs used in cryptographic programs. 
In order to build secure cryptosystems, we develop a certified
technique to verify low-level mathematical constructs used in the
security protocol X25519 automatically in this paper.

X25519 is an Elliptic Curve Diffie-Hellman (ECDH) key exchange
protocol; it is a high-performance cryptosystem designed to 
use the secure elliptic curve Curve25519~\cite{Ber06}. Curve25519 is an elliptic
curve offering 128 bits of security when used with ECDH. In addition
to allowing high-speed elliptic curve arithmetic, it is easier to
implement properly, not covered by any known patents, and moreover
less susceptible to implementation pitfalls such as weak 
random-number generators. Its parameters were also selected by
easily described mathematical principles.
% without resorting to any random numbers or seeds. 
These characteristics make Curve25519 a
preferred choice for those who are leery of curves which might have
intentionally inserted
backdoors, such as those standardized by the United States National
Institute of Standards and Technology (NIST). 
Indeed, Curve25519 is currently the
de facto alternative to the NIST P-256 curve. Consequently, X25519 has
a wide variety of applications including the default key exchange
protocol in \openssh since 2014~\cite{W:17:C}.

Most of the computation in X25519, in trade parlance, is in a
``variable base point multiplication,'' and the centerpiece 
is the Montgomery Ladderstep. This is usually a
large constant-time assembly program performing the
finite-field arithmetic that implements the mathematics on Curve25519.
Should the implementation of Montgomery Ladderstep be incorrect, so
would that of X25519. Obviously for all its virtues, X25519 would be
pointless if its implementation is incorrect. This may be even more
relevant in cryptography than most of engineering, because cryptography is
one of the few disciplines with the concept of an omnipresent
adversary, constantly looking for the smallest edge --- and hence
eager to trigger any unlikely event. Revising a cryptosystem
due to rare failures potentially leading to a cryptanalysis is not
unheard of~\cite{HNPPSSW:03:IDFSNE}.
Thus, it is important for security that we can show the computations
comprising the Montgomery Ladderstep or (even better) the X25519
protocol to be correct. 
% However, such verification is not easy due the size both of the
% numbers in play (255 bits and more) and of the program itself
% (10,000+ machine instructions).

Several obstacles need be overcome for the verification of mathematical
constructs in X25519. The key exchange protocol is based on a
group induced by Curve25519. The elliptic
curve is in turn defined over the Galois field $\bbfGF(2^{255} - 19)$. 
To compute on the elliptic curve group, arithmetic computation over
$\bbfGF(2^{255} - 19)$ needs to be correctly implemented. 
Particularly, 255-bit multiplications modulo
$2^{255} - 19$ must be verified. Worse, commodity computing devices do
not support 255-bit arithmetic computation directly. Arithmetic over
the Galois field needs to be implemented by sequences of 32- or 64-bit
instructions of the underlying architectures. One has to
verify that a sequence of 32- or 64-bit instructions indeed
computes, say, a 255-bit multiplication over the finite field. Yet this
is only a single step in the operation on the elliptic curve group.
In order to compute the group operation, another sequence of
arithmetic computation over $\bbfGF(2^{255} - 19)$ is
needed. Particularly, a crucial step,
the Montgomery Ladderstep, requires 18 arithmetic 
computations over $\bbfGF(2^{255} - 19)$~\cite{M:87:SPEC}. 
The entire Ladderstep must be
verified to ensure security guarantees offered by Curve25519.
% and hence the ECDH key exchange protocol X25519.

In this paper, we focus on algebraic properties about low-level
implementations of mathematical constructs in cryptographic programs.
Mathematical constructs by their nature perform computation on
underlying algebraic structures. We aim to verify whether they perform
intended algebraic computation correctly. To this end, we propose the
domain specific language \mydsl for low-level 
mathematical constructs. Algebraic pre- and post-conditions of
programs in \mydsl are specified as Hoare
triples~\cite{H:69:ABCP}. 
Such an algebraic specification is converted to static single 
assignment form and then translated into an algebraic problem (called 
the modular polynomial equation entailment
problem)~\cite{AWZ:88:DQVP,H:07:AENTP}. We use the computer
algebra system \singular to solve the algebraic problem~\cite{GP:08:SICA}. 
Program fragments irrelevant to algebraic properties are also
removed by slicing to reduce the size of algebraic problems~\cite{W:81:PS}.
The proof assistant \coq is used to certify the
correctness of translations, as well as solutions to algebraic
problems computed by \singular~\cite{YC:2004:ITPPDC}.

We report case studies on verifying mathematical constructs used in
the X25519 ECDH key exchange protocol~\cite{BDL+:11:HSHSS,BDL+:12:HSHSS}. 
For each arithmetic operations
(such as addition, subtraction, and multiplication) over $\bbfGF(2^{255} - 19)$,
their low-level real-world implementations are converted to our domain
specific language \mydsl manually. We specify algebraic properties
of mathematical constructs in Hoare triples. Mathematical constructs
are then verified against their algebraic
specifications with our automatic technique. 
The implementation of the Montgomery Ladderstep is 
verified similarly.  

We have the following contributions:
\begin{itemize}
\item We propose a domain specific language \mydsl for modeling low-level
  mathematical constructs used in cryptographic programs.
\item We give a certified verification condition generator from
  algebraic specifications of programs to the modular polynomial
  equation entailment problem.
\item We verify arithmetic computation over a finite field of order
  $2^{255} - 19$ and a
  critical program (the Montgomery Ladderstep) automatically.
\item To the best of our knowledge, our work is the first automatic
  and certified verification on real cryptographic programs with
  minimal human intervention.
\end{itemize}


\noindent
\emph{Related Work}.
Low-level implementations of mathematical constructs have been formalized and manually proved in proof assistants~\cite{Aff13,ANY12,AM07,MG07,M:13:PPVB}.
A semi-automatic approach~\cite{C:14:VCS} has successfully verified a hand-optimized assembly implementation of the Montgomery Ladderstep with SMT solvers, manual program annotation, and a few \coq proofs.
A C implementation of the Montgomery Ladderstep has been automatically verified with \gfverif~\cite{BS:16:GFEV}.
Re-implementations of mathematical constructs in F*~\cite{project:everest} have been verified using a combination of SMT solving and manual proofs.
Several cryptographic implementations in C and Java have been automatically verified by SAW to be equivalent to their reference implementations written in Cryptol~\cite{T:16:AVRWCI} but the correctness of reference implementations is not proven and the verification results are not certified.
The \openssl implementations of SHA-256 and HMAC have been formalized and manually proved in \coq~\cite{A:15:VCPS,B:15:VCSOH}.
Synthesis of assembly codes for mathematical constructs has been proposed in~\cite{fiat:crypto}.
Although the synthesized codes are correct by construction, they are not as efficient as hand-optimized assembly implementations.





\hide{
\cite{AM07} manually prove low-level implementations in SmartMIPS with \coq
\cite{ANY12}
\cite{AFF13} introduce a formalization of data structures for signed multi-precision arithmetic and we experiment it with formal verification of basic functions, using Separation logic
\cite{MG07} Hoare Logic, HOL4, ARM


One approach is to re-implement cryptographic protocols in languages and frameworks that allow efficient verification.
The most extensive work in this area is miTLS, a ``verified reference implementation of the TLS protocol''~\cite{B:14:VRTI,B:13:ITVCS}.
This implementation of TLS is written in F\# and specified in F7 -- the clear focus is on a verifiable (and verified)
re-implementation; not on verifying existing high-speed cryptographic software.
Note that miTLS relies on (unverified)
``cryptographic providers such as .NET or Bouncy Castle''
for core cryptographic primitives.
Also the \cryptver project~\cite{cryptver} aims at re-implementing cryptography such that it can be formally verified.
Their approach is to specify cryptographic algorithms in higher-order logic
and then implement them by formal deductive compilation.

Another approach to formally verified cryptographic software are special-domain compilers.
A recent example of this is~\cite{ABBD13},
where Almeida et al. introduce
\emph{security-aware compilation} of a subset of the C programming language.

The theory of elliptic curve has been formalized in~\cite{TH07,Bar11}.
In principle, the mathematical formulas in Curve25519 Diffie-Hellman
key-exchange can be verified with the formalization.
Low-level machine codes have been formalized in proof
assistants~\cite{Aff13,ANY12,AM07,MG07}. Large-integer arithmetic and
cryptographic functions can be formally verified semi-automatically.
Our approach is very lightweight. Most of the verification is
performed by an SMT solver automatically. It hence requires much less
human intervention.
}

%%% Local Variables:
%%% mode: latex
%%% eval: (TeX-PDF-mode 1)
%%% eval: (TeX-source-correlate-mode 1)
%%% TeX-master: "certified_vcg"
%%% End:



This paper is organized as follows. After preliminaries
(Section~\ref{section:preliminaries}), our domain specific
language is described in Section~\ref{section:domain-specific-language}. 
Section~\ref{section:translation}
presents the translation to the algebraic
problem. A certified solver for the algebraic problem is discussed in
Section~\ref{section:solving-algebraic-equations}. 
Section~\ref{section:evaluation} contains experimental results. It is
followed by conclusions.

%%% Local Variables: 
%%% mode: latex
%%% eval: (TeX-PDF-mode 1)
%%% eval: (TeX-source-correlate-mode 1)
%%% TeX-master: "certified_vcg"
%%% End: 
