
The last step transforms any algebraic program specification in \zdsl to the
modular polynomial equation entailment problem. For
$q \in \zPred$, we write $q(\vx)$ to signify the free variables $\vx$
occurred in $q$. Given $q(\vx), q'(\vx) \in \zPred$, the \emph{modular
  polynomial equation entailment} problem decides whether
$q(\vx) \implies q'(\vx)$ holds when $\vx$ are substituted for
arbitrary integers. That is, we want to check if for every valuation $\zst \in \zSt$, $q(\vx)$ evaluates to $\btt$ implies $q'(\vx)$ evaluates to $\btt$ after each variable $x$ is replaced by $\zst(x)$.
We write $\bbfZ \models \forall \vx. q(\vx) \implies q'(\vx)$ if it is indeed the case.

Programs in static single assignment form can easily be transformed to
conjunctions of polynomial equations. An assignment statement is
translated to a polynomial equation with a variable on the left hand side.
A $\dslcode{Split}$ statement is transformed to an
equation with a linear expression of the assigned variables on the
left hand side (Algorithm~\ref{algorithm:polynomial-statements}).
\begin{algorithm}
  \begin{algorithmic}[1]
    \Function{StmtToPolyEQ}{$s$}
    \Match{$s$}
      \Case{$v \leftarrow e$}
        \Return $v = e$
      \EndCase
      \Case{$\concat{v_h}{v_l} \leftarrow \dslcode{Split}(e, n)$}
        \State{\Return $v_l + \dslcode{Pow}(2, n) \times v_h = e$}
      \EndCase
    \EndMatch
    \EndFunction
  \end{algorithmic}
  \caption{Polynomial Equation Transformation for Statements}
  \label{algorithm:polynomial-statements}
\end{algorithm}

A program in static single assignment form is transformed to the
conjunction of polynomial equations corresponding to its statements
(Algorithm~\ref{algorithm:polynomial-programs}).

\begin{algorithm}
  \begin{algorithmic}[1]
    \Function{ProgToPolyEQ}{$p$}
    \Match{$p$}
      \Case{$\epsilon$} \Return $\top$ \EndCase
      \Case{$s; pp$}
        \State{\Return \Call{StmtToPolyEQ}{$s$} $\wedge$ \Call{ProgToPolyEQ}{$pp$}}
      \EndCase
    \EndMatch
    \EndFunction
  \end{algorithmic}
  \caption{Polynomial Equation Transformation for Programs}
  \label{algorithm:polynomial-programs}
\end{algorithm}

Algorithm~\ref{algorithm:polynomial-statements}
and~\ref{algorithm:polynomial-programs} are specified straightforwardly
in \gallina. We use the proof assistant \coq to prove properties
about the algorithms. Note that $\textsc{ProgToPolyEQ}(p) \in
\zPred$ for every $p \in \zProg$. The following theorem shows that any
behavior of the program $p$ is a solution to the system of polynomial
equations $\textsc{ProgToPolyEQ}(p)$. In other words,
$\textsc{ProgToPolyEQ}(p)$ gives an abstraction of the program $p$.

\begin{theorem}
  Let $p \in \zProg$ be a well-formed program in static single assignment
  form. For every $\zst, \zst' \in \zSt$ with $\zst \goesto{p} \zst'$,
  we have $\bbfZ \models \textsc{ProgToPolyEQ}(p) [\zst']$.
\end{theorem}

Definition~\ref{definition:program-entailment} gives the modular
polynomial equation entailment problem corresponding to an algebraic
program specification.
\begin{definition}
  For $q, q' \in \zPred$ and $p \in \zProg$ in static single assignment
  form, define
  \begin{eqnarray*}
    \Pi(\hoaretriple{q}{p}{q'}) & \defn &
    q(\vx) \wedge \varphi(\vx) \implies q'(\vx)
  \end{eqnarray*}
  where $\varphi(\vx) =
  \textsc{ProgToPolyEQ}(p)$.
  \label{definition:program-entailment}
\end{definition}

\begin{figure*}
  \centering
  $
  \begin{array}{l}
  \top \wedge
  \left(
  \begin{array}{lclcl}
    \begin{array}{rclc}
      {r}^1_0 & = & {x}^0_0 & \wedge \\
      {r}^1_1 & = & {x}^0_1 & \wedge \\
      {r}^1_2 & = & {x}^0_2 & \wedge \\
      {r}^1_3 & = & {x}^0_3 & \wedge \\
      {r}^1_4 & = & {x}^0_4 & \wedge \\
    \end{array}
    &\hspace{.01\textwidth}&
    \begin{array}{rclc}
      {r}^2_0 & = & {r}^1_0 + {4503599627370458} & \wedge \\
      {r}^2_1 & = & {r}^1_1 + {4503599627370494} & \wedge \\
      {r}^2_2 & = & {r}^1_2 + {4503599627370494} & \wedge \\
      {r}^2_3 & = & {r}^1_3 + {4503599627370494} & \wedge \\
      {r}^2_4 & = & {r}^1_4 + {4503599627370494} & \wedge\\
    \end{array}
    &\hspace{.01\textwidth}&
    \begin{array}{rclc}
      {r}^3_0 & = & {r}^2_0 - {y}^0_0 & \wedge \\
      {r}^3_1 & = & {r}^2_1 - {y}^0_1 & \wedge \\
      {r}^3_2 & = & {r}^2_2 - {y}^0_2 & \wedge \\
      {r}^3_3 & = & {r}^2_3 - {y}^0_3 & \wedge \\
      {r}^3_4 & = & {r}^2_4 - {y}^0_4
    \end{array}
  \end{array}
  \right) \implies \\
    %\hspace{.05\textwidth}
    \mathit{radix51}_\bbfZ({x}^0_4, {x}^0_3, {x}^0_2, {x}^0_1, {x}^0_0) -
    \mathit{radix51}_\bbfZ({y}^0_4, {y}^0_3, {y}^0_2, {y}^0_1, {y}^0_0)
    \equiv
    \mathit{radix51}_\bbfZ({r}^3_4, {r}^3_3, {r}^3_2, {r}^3_1, {r}^3_0)
    \mod \myprime
  \end{array}
  $
  \caption{Modular Polynomial Equation Entailment for \textsc{BV2ZProg}(\dslcode{bSubSSA})}
  \label{figure:translation:subtraction-polynomial}
\end{figure*}

\noindent
\emph{Example}.
The modular polynomial equation entailment problem corresponding to the
algebraic specification of subtraction is shown in
Figure~\ref{figure:translation:subtraction-polynomial}. The problem
has 15 polynomial equality constraints with 25 variables.
Define $\mathit{radix51}_\bbfZ({x}_4, {x}_3, {x}_2, {x}_1, {x}_0) \defn \dslcode{Pow}(2, 51 \times_\bbfZ 4) \times {x}_4 + \dslcode{Pow}(2, 51 \times_\bbfZ 3) \times {x}_3 + \dslcode{Pow}(2, 51 \times_\bbfZ 2) \times {x}_2 + \dslcode{Pow}(2, 51 \times_\bbfZ 1) \times {x}_1 + \dslcode{Pow}(2, 51 \times_\bbfZ 0) \times {x}_0$ for ${x}_0, {x}_1, {x}_2, {x}_3, {x}_4 \in \Var$.
We want to know if $\mathit{radix51}_\bbfZ({r}^3_4, {r}^3_3, {r}^3_2,
{r}^3_1, {r}^3_0)$ is the difference between $\mathit{radix51}_\bbfZ({x}^0_4,
{x}^0_3, {x}^0_2, {x}^0_1, {x}^0_0)$ and $\mathit{radix51}_\bbfZ({y}^0_4,
{y}^0_3, {y}^0_2, {y}^0_1, {y}^0_0)$ in $\bbfGF(\varrho)$ under the
constraints.

The soundness of Algorithm~\ref{algorithm:polynomial-programs}
is certified in \coq (Theorem~\ref{theorem:program-to-q-soundness}).
% Its proof is again certified by \coq.
\begin{theorem}
  \label{theorem:program-to-q-soundness}
  Let $q, q' \in \zPred$ be predicates, and $p \in \zProg$ a well-formed
  program in static single assignment form.
  If $\bbfZ \models \forall \vx.\Pi(\hoaretriple{q}{p}{q'})$, then
  $\models$ $\hoaretriple{q}{p}{q'}$.
\end{theorem}
