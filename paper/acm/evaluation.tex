
We evaluate our techniques in real-world low-level mathematical
constructs in X25519. 
In elliptic curve cryptography, arithmetic computation over
large finite fields is required. For
instance, Curve25519 defined by $y^2 = x^3 + 486662 x^2 + x$ 
is over the Galois field $\bbfF = \bbfGF(\varrho)$ with
$\varrho = 2^{255} - 19$. To
make the field explicit, we rewrite its definition as:
\vspace{-.5em}
\begin{equation}
  \label{eq:curve25519}
  y \Ftimes y \Feq x \Ftimes x \Ftimes x \Fplus
  486662 \Ftimes x \Ftimes x \Fplus x.
\end{equation}

Since arithmetic computation is over $\bbfF$ whose
elements can be represented by 255-bit numbers,
any pair $(x, y)$ satisfying (\ref{eq:curve25519}) (called a
\emph{point} on the curve) can be represented by a pair of 255-bit
numbers. It can be shown that points on Curve25519 with the point at
infinity as the unit (denoted $\Gzero$) form a commutative group $\bbfG = (G, \Gplus, \Gzero)$
with $G = \{ (x, y) : x, y \textmd{ satisfying } (\ref{eq:curve25519})
\}$. Let $P_0 = (x_0, y_0), P_1 = (x_1, y_1) \in G$. We have $-P_0 =
(x_0, -y_0)$ and $P_0 \Gplus P_1 = (x, y)$  where
\vspace{-.5em}
\begin{eqnarray}
  m & = & (y_1 \Fminus y_0) \Fdiv (x_1 \Fminus x_0)\\\nonumber
  x & = & m \Ftimes m \Fminus 486662 \Fminus x_0 \Fminus x_1\\ \nonumber
  y & = & %m \Ftimes (x_0 \Fminus x) \Fminus y_0 
     %= m \Ftimes (x_1 \Fminus x) \Fminus y_1.
          (2 \Ftimes x_0 \Fplus x_1 \Fplus 486662 )\Ftimes m
          \Fminus m \Ftimes m \Ftimes m\Fminus y_0
  \label{eq:curve25519-group}
\end{eqnarray}
when $P_0 \neq \pm P_1$. Other cases ($P_0 = \pm P_1$) are defined
similarly~\cite{C:96:CCANT}.
$\bbfG$ and similar elliptic curve groups
are the main objects in elliptic curve cryptography. It is
essential to implement the commutative binary operation $\Gplus$ very
efficiently in practice.
\hide{
When $P_0=-P_1$, the sum is the point at infinity; When $P_0= P_1$,
$P_0 \Gplus P_1 = (x, y)$ where
\begin{eqnarray}
  \label{eq:curve25519-dbl}
  m &=& (3  \Ftimes x_1 \Ftimes x_1 \Fplus 2\Ftimes a\Ftimes x_1\Fplus 1) 
        \Fdiv (2\Ftimes b\Ftimes y_1) \\ \nonumber
   x &=& b \Ftimes m \Ftimes m \Fminus a \Fminus 2 \Ftimes x_1 \\ \nonumber
   y &=& (3 \Ftimes x1\Fplus a) \Ftimes m \Fminus b \Ftimes m \Ftimes m \Ftimes m\Fminus y_1
\end{eqnarray}
%
}

%%% Local Variables: 
%%% mode: latex
%%% eval: (TeX-PDF-mode 1)
%%% eval: (TeX-source-correlate-mode 1)
%%% TeX-master: "certified_vcg"
%%% End: 
