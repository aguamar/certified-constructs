
The last step transforms any algebraic program specification to the 
modular polynomial equation entailment problem. For
$q \in \Pred$, we write $q(\vx)$ to signify the free variables $\vx$
occurred in $q$. Given $q(\vx), q'(\vx) \in \Pred$, the \emph{modular
  polynomial equation entailment} problem decides whether 
$q(\vx) \implies q'(\vx)$ holds when $\vx$ are substituted for
arbitrary integers. That is, we want to check if $q(\vx) \implies
q'(\vx)$ evaluates to $\btt$ after each variable $x$ is replaced by
$\nu(x)$ for every valuation $\nu \in \St$. We write $\bbfZ \models
\forall \vx. q(\vx) \implies q'(\vx)$ if it is indeed the case.

Programs in static single assignment form can easily be transformed to
conjunctions of polynomial equations. An assignment statement is
translated to a polynomial equation with a single variable on the left
hand side. For a $\dslcode{Split}$ statement, it is transformed to an
equation with a linear expression of the assigned variables on the
left hand side (Algorithm~\ref{algorithm:polynomial-statements}). 
\begin{algorithm}
  \begin{algorithmic}[1]
    \Function{StmtToPolyEQ}{$s$}
    \Match{$s$}
      \Case{$v \leftarrow e$}
        \Return $v = e$
      \EndCase
      \Case{$\concat{v_h}{v_l} \leftarrow \dslcode{Split}(e, n)$}
        \Return $v_l + 2^n v_h = e$
      \EndCase
    \EndMatch
    \EndFunction
  \end{algorithmic}
  \caption{Polynomial Equation Transformation for Statements}
  \label{algorithm:polynomial-statements}
\end{algorithm}

A program in static single assignment form is transformed to the
conjunction of polynomial equations corresponding to its statements
(Algorithm~\ref{algorithm:polynomial-programs}). 

\begin{algorithm}
  \begin{algorithmic}[1]
    \Function{ProgToPolyEQ}{$p$}
    \Match{$p$}
      \Case{$\epsilon$} \Return $\top$ \EndCase
      \Case{$s; pp$}
        \Return \Call{StmtToPolyEQ}{$s$} $\wedge$
                \Call{ProgToPolyEQ}{$pp$}
      \EndCase
    \EndMatch
    \EndFunction
  \end{algorithmic}
  \caption{Polynomial Equation Transformation for Programs}
  \label{algorithm:polynomial-programs}
\end{algorithm}

Algorithm~\ref{algorithm:polynomial-statements}
and~\ref{algorithm:polynomial-programs} are specified straightforwardly 
in \gallina. We use the proof assistant \coq to prove properties
about the algorithms. First, note that $\textsc{ProgToPolyEQ}(p) \in
\Pred$ for every $p \in \Prog$. The following theorem shows that any
behavior of the program $p$ is a solution to the system of polynomial
equations $\textsc{ProgToPolyEQ}(p)$. In other words,
$\textsc{ProgToPolyEQ}(p)$ gives an abstraction of the program $p$.

\begin{theorem}
  Let $p \in \Prog$ be a well-formed program in static single assignment
  form. For every $\nu, \nu' \in \St$ with $\nu \goesto{p} \nu'$, 
  we have $\bbfZ \models \textsc{ProgToPolyEQ}(p) [\nu']$.
\end{theorem}

Definition~\ref{definition:program-entailment} gives the modular
polynomial equation entailment problem corresponding to an algebraic
program specification.
\begin{definition}
  For $q, q' \in \Pred$ and $p \in \Prog$ in static single assignment
  form, define
  \begin{eqnarray*}
    \Pi(\hoaretriple{q}{p}{q'}) & \defn &
    q(\vx) \wedge \varphi(\vx) \implies q'(\vx)
  \end{eqnarray*}
  where $\varphi(\vx) =
  \textsc{ProgramToPolyEQ}(p)$. 
  \label{definition:program-entailment}
\end{definition}

\vspace{-2em}
\begin{figure}
  \centering
  \[
  \begin{array}{l}
  \top \wedge
  \left(
  \begin{array}{lclcl}
    \begin{array}{rclc}
      {r}^0_0 & = & {x}^0_0 & \wedge \\
      {r}^0_1 & = & {x}^0_1 & \wedge \\
      {r}^0_2 & = & {x}^0_2 & \wedge \\
      {r}^0_3 & = & {x}^0_3 & \wedge \\
      {r}^0_4 & = & {x}^0_4 & \wedge \\
    \end{array}
    &\hspace{.01\textwidth}&
    \begin{array}{rclc}
      {r}^1_0 & = & {r}^0_0 + {4503599627370458} & \wedge \\
      {r}^1_1 & = & {r}^0_1 + {4503599627370494} & \wedge \\
      {r}^1_2 & = & {r}^0_2 + {4503599627370494} & \wedge \\
      {r}^1_3 & = & {r}^0_3 + {4503599627370494} & \wedge \\
      {r}^1_4 & = & {r}^0_4 + {4503599627370494} & \wedge\\
    \end{array}
    &\hspace{.01\textwidth}&
    \begin{array}{rclc}
      {r}^2_0 & = & {r}^1_0 - {y}^0_0 & \wedge \\
      {r}^2_1 & = & {r}^1_1 - {y}^0_1 & \wedge \\
      {r}^2_2 & = & {r}^1_2 - {y}^0_2 & \wedge \\
      {r}^2_3 & = & {r}^1_3 - {y}^0_3 & \wedge \\
      {r}^2_4 & = & {r}^1_4 - {y}^0_4
    \end{array}
  \end{array}
  \right)\\
    \hspace{.05\textwidth}
    \implies 
    \mathit{radix}({x}^0_4, {x}^0_3, {x}^0_2, {x}^0_1, {x}^0_0) -
    \mathit{radix51}({y}^0_4, {y}^0_3, {y}^0_2, {y}^0_1, {y}^0_0)
    \equiv
    \mathit{radix51}({r}^2_4, {r}^2_3, {r}^2_2, {r}^2_1, {r}^2_0)
    \mod \myprime
  \end{array}
  \]
  \caption{Modular Polynomial Equation Entailment for \dslcode{subSSA}}
  \label{figure:translation:subtraction-polynomial}
\end{figure}
\vspace{-1em}

\noindent
\emph{Example}.
The modular polynomial equation entailment problem coresponding to the
algebraic specification of subtraction is shown in
Figure~\ref{figure:translation:subtraction-polynomial}. The problem
has 15 polynomial equality constraints with 25 variables. We 
want to know if $\mathit{radix51}({r}^2_4, {r}^2_3, {r}^2_2,
{r}^2_1, {r}^2_0)$ is the difference between $\mathit{radix}({x}^0_4,
{x}^0_3, {x}^0_2, {x}^0_1, {x}^0_0)$ and $\mathit{radix51}({y}^0_4,
{y}^0_3, {y}^0_2, {y}^0_1, {y}^0_0)$ in $\bbfGF(\varrho)$ under the
constraints. 

The soundnesss of Algorithm~\ref{algorithm:polynomial-programs}
is certified in \coq (Theorem~\ref{theorem:program-to-q-soundness}).
% Its proof is again certified by \coq.
\begin{theorem}
  \label{theorem:program-to-q-soundness}
  Let $q, q' \in \Pred$ be predicates, and $p \in \Prog$ a program in
  static single assignment form and well-formed. 
  If $\bbfZ \models \forall \vx.\Pi(\hoaretriple{q}{p}{q'})$, then
  $\models$ $\hoaretriple{q}{p}{q'}$.
\end{theorem}

