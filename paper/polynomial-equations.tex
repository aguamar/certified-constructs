
The last step transforms any program specification to the 
modular polynomial equation entailment problem defined as follows. For
$q \in Q$, we write $q(\vx)$ to signify the free variables $\vx$
occurred in $q$. Given $q(\vx), q'(\vx) \in Q$, the \emph{modular
  polynomial equation entailment} problem decides whether $\forall
\vx. q(\vx) \implies q'(\vx)$ holds for all $\vx \in \bbfZ^{|\vx|}$.

Programs in single static assignment form can easily be transformed to
conjunctions of polynomial equations. We first show how to transform
statements into polynomial equations. An assignment statement is
translated to a polynomial equation with a single variable on the left
hand side. For a $\dslcode{Split}$ statement, it is transformed to an
equation with a linear expression of the assigned variables on the
left hand side (Algorithm~\ref{algorithm:polynomial-statements}). 
\begin{algorithm}
  \begin{algorithmic}[1]
    \Function{StatementToPolyEQ}{$s$}
    \Match{$s$}
      \Case{$v \leftarrow e$}
        \Return $v = e$
      \EndCase
      \Case{$\concat{v_h}{v_l} \leftarrow \dslcode{Split}(e, n)$}
        \Return $v_l + 2^n v_h = e$
      \EndCase
    \EndMatch
    \EndFunction
  \end{algorithmic}
  \caption{Polynomial Equation Transformation for Statements}
  \label{algorithm:polynomial-statements}
\end{algorithm}

A program is transformed to the conjunction of polynomial
equations corresponding to its statements
(Algorithm~\ref{algorithm:polynomial-programs}). Both algorithms are
specified straightforwardly in \coq.

\begin{algorithm}
  \begin{algorithmic}[1]
    \Function{ProgramToPolyEQ}{$p$}
    \Match{$p$}
      \Case{$\epsilon$} \Return $\top$ \EndCase
      \Case{$s; pp$}
        \Return \Call{StatementToPolyEQ}{$s$} $\wedge$
                \Call{ProgramToPolyEQ}{$pp$}
      \EndCase
    \EndMatch
    \EndFunction
  \end{algorithmic}
  \caption{Polynomial Equation Transformation for Programs}
  \label{algorithm:polynomial-programs}
\end{algorithm}



\begin{theorem}
  Let $p \in P$ be a well-formed program in static single assignment
  form. For every $\nu, \nu' \in \St$ with $\nu \goesto{p} \nu'$, 
  we have $\bbfZ \models \textsc{ProgramToPolyEQ}(p) [\nu']$.
\end{theorem}


Note that $\textsc{ProgramToPolyEQ}(p) \in Q$ for every $p \in P$.
Definition~\ref{definition:program-entailment} gives the modular
polynomial equation entailment problem corresponding to a program
specification.
\begin{definition}
  For any $q, q' \in Q$ and $p \in P$ in single static assignment
  form, define
  \begin{eqnarray*}
    \Pi(\hoaretriple{q}{p}{q'}) & \defn &
    \forall \vx. q(\vx) \wedge \varphi(\vx) \implies q'(\vx)
  \end{eqnarray*}
  where $\varphi(\vx) =
  \textsc{ProgramToPolyEQ}(p)$. 
  %$\Pi(\hoaretriple{q}{p}{q'})$ is called the modular polynomial
  %equation entailment problem for $\hoaretriple{q}{p}{q'}$. 
  \label{definition:program-entailment}
\end{definition}

\begin{figure}
  \centering
  \[
  \begin{array}{l}
  \forall 
     {x}^0_0, {x}^0_1, {x}^0_2,
     {x}^0_3, {x}^0_4,
     {y}^0_0, {y}^0_1, {y}^0_2,
     {y}^0_3, {y}^0_4,
     {r}^0_0, {r}^0_1, {r}^0_2,
     {r}^0_3, {r}^0_4,
     {r}^1_0, {r}^1_1, {r}^1_2,
     {r}^1_3, {r}^1_4,
     {r}^2_0, {r}^2_1, {r}^2_2,
     {r}^2_3, {r}^2_4).\\
  \hspace{.05\textwidth}\top \wedge
  \left(
  \begin{array}{lclcl}
    \begin{array}{rclc}
      {r}^0_0 & = & {x}^0_0 & \wedge \\
      {r}^0_1 & = & {x}^0_1 & \wedge \\
      {r}^0_2 & = & {x}^0_2 & \wedge \\
      {r}^0_3 & = & {x}^0_3 & \wedge \\
      {r}^0_4 & = & {x}^0_4 & \wedge \\
    \end{array}
    &\hspace{.01\textwidth}&
    \begin{array}{rclc}
      {r}^1_0 & = & {r}^0_0 + {4503599627370458} & \wedge \\
      {r}^1_1 & = & {r}^0_1 + {4503599627370494} & \wedge \\
      {r}^1_2 & = & {r}^0_2 + {4503599627370494} & \wedge \\
      {r}^1_3 & = & {r}^0_3 + {4503599627370494} & \wedge \\
      {r}^1_4 & = & {r}^0_4 + {4503599627370494} & \wedge\\
    \end{array}
    &\hspace{.01\textwidth}&
    \begin{array}{rclc}
      {r}^2_0 & = & {r}^1_0 - {y}^0_0 & \wedge \\
      {r}^2_1 & = & {r}^1_1 - {y}^0_1 & \wedge \\
      {r}^2_2 & = & {r}^1_2 - {y}^0_2 & \wedge \\
      {r}^2_3 & = & {r}^1_3 - {y}^0_3 & \wedge \\
      {r}^2_4 & = & {r}^1_4 - {y}^0_4
    \end{array}
  \end{array}
  \right)\\
    \hspace{.05\textwidth}
    \implies 
    \mathit{radix}({x}^0_4, {x}^0_3, {x}^0_2, {x}^0_1, {x}^0_0) -
    \mathit{radix51}({y}^0_4, {y}^0_3, {y}^0_2, {y}^0_1, {y}^0_0)
    \equiv
    \mathit{radix51}({r}^2_4, {r}^2_3, {r}^2_2, {r}^2_1, {r}^2_0)
    \mod \myprime
  \end{array}
  \]
  \caption{Modular Polynomial Equation Entailment for \dslcode{subSSA}}
  \label{figure:translation:subtraction-polynomial}
\end{figure}

\noindent
\emph{Example}.
The modular polynomial equation entailment problem coresponding to the
algebraic specification of subtraction is shown in
Figure~\ref{figure:translation:subtraction-polynomial}. The problem
has 15 polynomial constraints with 25 integer variables. We would like
to see if $\mathit{radix51}({r}^2_4, {r}^2_3, {r}^2_2,
{r}^2_1, {r}^2_0)$ is the difference between $\mathit{radix}({x}^0_4,
{x}^0_3, {x}^0_2, {x}^0_1, {x}^0_0)$ and $\mathit{radix51}({y}^0_4,
{y}^0_3, {y}^0_2, {y}^0_1, {y}^0_0)$ in $\bbfGF(\varrho)$ under the
constraints. 



Algorithm~\ref{algorithm:polynomial-programs} is only sound for
well-formed programs. 
Theorem~\ref{theorem:program-to-q-soundness} establishes the
soundnesss of our last transformation. Its proof is also certified in
\coq. 
\begin{theorem}
  \label{theorem:program-to-q-soundness}
  Let $q, q' \in Q$ be predicates, and $p \in P$ a program in single
  static assignment form and well-formed. 
  If $\bbfZ \models \Pi(\hoaretriple{q}{p}{q'})$, then
  $\models$ $\hoaretriple{q}{p}{q'}$.
\end{theorem}

