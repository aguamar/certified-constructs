
Recall that X25519 is based on the Abelian group $\bbfG = (G, \Gplus,
\Gzero)$ induced by the curve Curve25519. As aforementioned, the binary
operation $\Gplus$ requires another sequence of arithmetic computation
over $\bbfGF(\varrho)$. Errors could still be introduced or even
implanted in any sequence of computation proclaimed to implement $\Gplus$.
Our next experiment verifies a critical low-level program implementing
the group operation~\cite{BDL+:11:HSHSS,BDL+:12:HSHSS}.

Let $P \in G$ be a point on Curve25519. We write $[n]P$ for the
$n$-fold addition ${P \Gplus \cdots \Gplus P} \in G$ for $n \in \bbfN$.
In X25519, we want to compute a \emph{point multiplication}, that is,
the point $[n]P$ for given $n$ and $P$. The standard iterative
squaring method computes $[n]P$ by examining each bit of $n$ iteratively.
For each iteration, $[2m]P$ is computed from $[m]P$ and
added with another $P$ when the current bit is $1$. Although the
method is reasonably efficient, it is not constant-time and hence insecure. 

\hide{
The operation we want is a ``point multiplication'' which is given $n$
and $P$ compute $[n]P$.  The na\"ive way to do this is ``double
and add''.  
\begin{algorithm}[h]
  \begin{algorithmic}[1]  
    \Function{point.multiple}{$m,P$}
    \begin{multicols}{2}
      \State $t\leftarrow P$
      \For{$i=\lfloor \log_2 m \rfloor-1\ \mathbf{downto}\ 1$}
      \State $t\leftarrow [2]T$
      \If{$($ bit $i$ of $n$ in binary $)=1$}
      \State $t\leftarrow t\Gplus P$
      \EndIf
      \EndFor
      \State \Return{$t$}
      \EndFunction
    \end{multicols}
  \end{algorithmic}
\end{algorithm}
Observe that given $[n]P$ before step 4, we first compute $[2n]P$
or $[2n+1]P$ by the end of step 7 by first doubling to
$[2n]P$ (step 4), then decide to add $P$ or not depending on a
whether a corresponding bit is 1.  But this is obviously not time
constant.  The obvious next step that we can double
(i.e. compute $[2n]P$), \emph{always add} $P$, then choose one
of the two using conditional moves or equivalents.  This however is
inefficient.  We need something equivalent to steps 4--7 but better.
}

\vspace{-.5em}
\begin{algorithm}[h]
\begin{algorithmic}[1]
\Function{Ladderstep}{$x_1, x_m, z_m, x_{m+1}, z_{m+1}$}
\begin{multicols}{2}
\State $t_1 \leftarrow x_m \Fplus z_m$
\State $t_2 \leftarrow x_m \Fminus z_m$
\State $t_7 \leftarrow t_2 \Ftimes t_2$
\State $t_6 \leftarrow t_1 \Ftimes t_1$
\State $t_5 \leftarrow t_6 \Fminus t_7$
\State $t_3 \leftarrow x_{m+1} \Fplus z_{m+1}$
\State $t_4 \leftarrow x_{m+1} \Fminus z_{m+1}$
\State $t_9 \leftarrow t_3 \Ftimes t_2$
\State $t_8 \leftarrow t_4 \Ftimes t_1$
\State $x_{m+1} \leftarrow t_8 \Fplus t_9$
\State $z_{m+1} \leftarrow t_8 \Fminus t_9$
\State $x_{m+1} \leftarrow x_{m+1} \Ftimes x_{m+1}$
\State $z_{m+1} \leftarrow z_{m+1} \Ftimes z_{m+1}$
\State $z_{m+1} \leftarrow z_{m+1} \Ftimes x_1$
\State $x_m \leftarrow t_6 \Ftimes t_7$
\State $z_m \leftarrow 121666 \Ftimes t_5$
\State $z_m \leftarrow z_m \Fplus t_7$
\State $z_m \leftarrow z_m \Ftimes t_5$
\State \Return $(x_m, z_m, x_{m+1}, z_{m+1})$
\EndFunction
\end{multicols}
\end{algorithmic}
\caption{Montgomery Ladderstep}
\label{algorithm:ladderstep}
\end{algorithm}
\vspace{-.5em}

To have constant execution time, the key idea is to compute \emph{both}
$[2m]P$ and $[2m+1]P$ at each iteration. The Montgomery Ladderstep is
an efficient algorithm computing $[2m]P$ and $[2m+1]P$ from $P$,
$[m]P$, and $[m+1]P$ on Montgomery curves (including Curve25519).
The algorithm uses only $x$ coordinates of the points. Furthermore,
expensive divisions are avoided in the Ladderstep by projective
representations. That is, the algorithm represents $x \Fdiv z$ by the
pair $x : z$ and works with fractions (Algorithm~\ref{algorithm:ladderstep}).

\hide{
Peter Montgomery derived a sequence of computations known
as the Montgomery Ladderstep, to compute both $P_0+P_1$ and $[2]P_0$
simultaneously and efficiently knowing $P_1-P_0$.  The Ladderstep can
be achieved only using $x$ coordinates on a set of curves today known as
Montgomery curves, which include Curve25519.  Furthermore, we can
avoid expensive divisions by using projective representations (i.e,
instead of $x$ write $x/z$ and work  with fractions).  

The Ladderstep starts from points $[n]P $ and
$[n+1]P $, to get either $ [2n]P $ and
$[2n+1]P$, or $[2n+1]P$ and $[2n+2]P$, the difference between the two points remaining $\pm P$.  Provided that there is a time-constant way to swap the two input
points, this
is equivalent to computing either $[2n]P$ or $[2n+1]P$
from $[n]P$ in constant
time and is sufficient to compute any $[n]P$ where
$n$ has a known bitlength.  This is known as a differential
addition chain and is one of the fundamental ways to avoid timing
attacks.  For curves not equivalent to a Montgomery curve, a variant
known as the "Co-$Z$ ladder" is available to effect the same
differential addition chain.
}

Let unprimed and primed variables denote their values before and after
computation respectively.
The Montgomery Ladderstep has the following
specification~\cite{M:87:SPEC}:\footnote{Amusingly, we find for
  ourselves the factor of $4$ in both the numerator and denominator of
  the addition formulas during verification, noted on
  \cite[p.~261]{M:87:SPEC}.}
%\vspace{-.5em}
\[
\begin{array}{rcl}
  x'_{m} & \Feq & 4 \Ftimes (x_m \Ftimes x_{m+1} \Fminus z_m \Ftimes z_{m+1})
               \Ftimes (x_m \Ftimes x_{m+1} \Fminus z_m \Ftimes z_{m+1}) \\
  z'_{m} & \Feq & 4 \Ftimes x_1 \Ftimes (x_m \Ftimes z_{m+1} \Fminus
               z_m \Ftimes x_{m+1})
               \Ftimes (x_m \Ftimes z_{m+1} \Fminus z_m \Ftimes x_{m+1}) \\
  x'_{m+1} & \Feq & (x_m \Ftimes x_m \Fminus z_m \Ftimes z_m) \Ftimes
                 (x_m \Ftimes x_m \Fminus z_m \Ftimes z_m) \\
  z'_{m+1} & \Feq & 4 \Ftimes x_m \Ftimes z_m \Ftimes
                 (x_m \Ftimes x_m \Fplus 121666 \Ftimes x_m \Ftimes z_m \Fplus
                 z_m \Ftimes z_m)
\end{array}
%\vspace{-.5em}
\]
In our experiment, we replace all arithmetic computation over $\bbfF$
with corresponding mathematical constructs (4 additions, 4 subtractions, 4 squares, 5
multiplications, and 1 multiplication by 121666) written in \mydsl,
translate the above specification into an algebraic specification, and
apply our technique to verify the entire Ladderstep.
The verification of the Ladderstep (containing 1462 statements) is carried out in 82 hours.
The SMT solver finishes range checks on the Ladderstep in 19 minutes.


\hide{
Montgomery's formulas and his steps are well known
since~\cite{M:87:SPEC}.  What we showed is that the program carries out
a computation equivalent to his formulas\footnote{An amusing fact is
  that when we carry out the Ladderstep, we find for ourselves the factor of $4$
  in both the numerator and denominator of the addition formulas, 
  noted on \cite[p.~261]{M:87:SPEC}.}.
}

% A
%crucial mathematical construct in X25519 is to compute the
%$x$-coordinate of $n \cdot p$ for $n \in \bbfN$. The na\"ive
%implementation would follow the equation (\ref{eq:curve25519-group})
%and compute both $x$- and $y$-coordinates with, in particular, a
%division operation $\Fdiv$ in $\bbfGF(\varrho)$. ,
%a more efficient implementation is proposed. 
%
%$np = x_n \Fdiv z_n$.
%$\textsc{ladderstep} (x_1, x_m, z_m, x_n, z_n) = (x_{2m}, z_{2m},
%x_{m+n}, z_{m+n})$
% 
%\todo{mention the coefficient $4$ in the ratio}
%

%%% Local Variables: 
%%% mode: latex
%%% eval: (TeX-PDF-mode 1)
%%% eval: (TeX-source-correlate-mode 1)
%%% TeX-master: "certified_vcg"
%%% End: 
