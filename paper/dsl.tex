
One of the big issues with modern cryptography is how the assumptions
match up with reality. In many situations, unexpected channels
through which information can leak to the attacker may cause the
cryptosystem to be broken. Typically this is about timing or
electric power used. 
In side-channel resilient implementations, the execution time
are kept constant (as much as possible) to prevent unexpected
information leakage.
%Implementations which are not defeated through
%such means are called ``side-channel resilient'' implementations.
Constant execution time however is harder to achieve than one would imagine.
%This turns out in some cases to be harder than one would imagine.
%This is chiefly because 
Modern processors have caches 
%and out of order execution 
and multitasking. This makes it possible for one execution
thread, even when no privilege is conferred, to affect the running
time of another -- simply by caching a sufficient amount of its own
data in correct locations through repeated accesses, and then
observing the running time of the other thread. The instructions in
the other thread which uses the ``evicted'' data (to make room for the
data of the eavesdropping thread) then has to take more time getting
its data back to the cache~\cite{B:05:CTAA}. 
% Such attacks are quite practical.
% \todo{insert reference in bib for D.J. Bernstein ``Cache Timing 
% Attacks on AES'', https://cr.yp.to/antiforgery/cachetiming-20050414.pdf}


\hide{
As such, modern cryptographic implementations often have to resort to
seemingly contrived sequences of conditional moves and arithmetic
manipulations to achieve the same result as simply using the CPU
instructions designed for the purpose. This of course slows the
computation down and waste resources but is inevitable because
secret-dependent conditional actions and table lookups are actually
very useful tasks that sometimes simply must be performed.
}

Thus, the innocuous actions of executing (a) a conditional
branch instruction dependent on a secret bit, and (b) an
indirect load instruction using a secret value in the register as the
address, are both potentially dangerous leaks of information.
%There are some side results of the above, 
Consequently, we are not often
faced with secret-dependent branching or table-lookups in the assembly
instructions, but a language describing cryptographic code might
include pseudo-instructions to cover instruction sequences, phrases in
the language if you will, that is used to achieve the same effect.
The domain specific language \mydsl is designed based on the same
principles. Conditional branches and indirect memory accesses are not
admitted in \mydsl. A program is but a straight line of variable
assignments on expressions. 
Consider the following syntactic classes:
\[
\begin{array}{rclcrclcrcl}
  \Nat & ::= & {1}\ |\ {2}\ |\ \cdots
  & \hspace{.02\columnwidth} &
  \Int & ::= & \cdots \ |\ {-2}\ |\ {-1} \ |\ 0\ |\ 
             {1}\ |\ {2}\ |\ \cdots
  & \hspace{.02\columnwidth} &
  \Var & ::= & {x} \ |\ {y} \ |\ {z} \ |\ \cdots\\
  \Expr & ::= &  \multicolumn{9}{l}
               { \Int \ |\ \Var \ |\  {-}\Expr \ |\ 
                 \Expr {+} \Expr \ |\ \Expr {-} \Expr \ |\ 
                 \Expr {\times} \Expr \ |\ \dslcode{Pow} (\Expr, \Nat) }
\end{array}
\]

We allow exact integers as constants in the domain specific
language. Variables are thus integer variables. An expression can be a
constant, a variable, or a negative expression. Additions, subtractions,
and multiplications of expressions are available. The expression
$\dslcode{Pow}(e, n)$ denotes $e^n$ for any expression $e$ and positive
integer $n$. More formally, let $\St \defn \Var \rightarrow
\bbfZ$ and $\nu \in \St$ be a \emph{state} (or \emph{valuation}). That is,
a {state} $\nu$ is a mapping from variables to integers. Define the
semantic function $\valueof{e}_{\nu}$ as follows.
\[
\begin{array}{rclcrcl}
  \valueof{{i}}_{\nu} & \defn & 
  \multicolumn{3}{c}{
    \begin{array}{lcccr}
      i \textmd{  for ${i} \in \Int$} 
      & \hspace{.02\columnwidth} &
      \valueof{{v}}_{\nu} \defn \nu({v}) 
      \textmd{  for ${v} \in \Var$}
      & \hspace{.02\columnwidth} &
      \valueof{-e}_{\nu} 
    \end{array}
  }
  & \defn & -_{\bbfZ}\valueof{e}_{\nu}
  \\
  \valueof{e_0 + e_1}_{\nu} & \defn & 
     \valueof{e_0}_{\nu} +_{\bbfZ} \valueof{e_1}_{\nu}
  & \hspace{.1\columnwidth} &
  \valueof{e_0 - e_1}_{\nu} & \defn & 
     \valueof{e_0}_{\nu} -_{\bbfZ} \valueof{e_1}_{\nu}\\
  \valueof{e_0 \times e_1}_{\nu} & \defn & 
     \valueof{e_0}_{\nu} \times_{\bbfZ} \valueof{e_1}_{\nu}
  &&
  \valueof{\dslcode{Pow}(e, n)}_{\nu} & \defn & 
     (\valueof{e}_{\nu})^{\valueof{n}_{\nu}}
\end{array}
\]

Note that there is no expression for quotients, or bitwise logical
operations. Bitwise left shifting however can be modeled by
multiplying $\dslcode{Pow}(2, n)$. Although \mydsl models a (very) 
small subset of assembly, it suffices to encode low-level
mathematical constructs in X25519.
\begin{eqnarray*}
  \Stmt & ::= & \Var \leftarrow \Expr
            \ |\  \concat{\Var}{\Var} \leftarrow \dslcode{Split} (\Expr, \Nat)
\end{eqnarray*}
In \mydsl, only assignments are allowed.
The statement $v \leftarrow e$ assigns the value of $e$
to the variable $v$. For bounded additions, multiplications, and right
shifting, they are modeled by the construct $\dslcode{Split}$ in \mydsl.
The statement $\concat{v_h}{v_l} \leftarrow
\dslcode{Split}(e, n)$ splits the value of $e$ into two parts;
the lowest $n$ bits are stored in $v_l$ and the remaining higher bits
are stored in $v_h$. Define
$
\nu\mymapsto{v}{d}(u) \defn
\left\{
   \begin{array}{ll}
     d & \textmd{if $u = v$}\\
     \nu(u) & \textmd{otherwise}
   \end{array}
\right.
$.
Consider the relation $\Tr \subseteq \St \times \Stmt \times \St$ defined
by $(\nu, v \leftarrow e, \nu\mymapsto{v}{\valueof{e}_{\nu}}) \in \Tr$, and
  $(\nu, \concat{v_h}{v_l} \leftarrow \dslcode{Split} (e, n), 
  \nu\mymapsto{v_h}{\mathit{hi}}\mymapsto{v_l}{\mathit{lo}}) \in \Tr$
where
$\mathit{hi} = (\valueof{e}_{\nu} - lo) \div 2^{\valueof{n}_{\nu}}$ and
$\mathit{lo} = \valueof{e}_{\nu} \mod 2^{\valueof{n}_{\nu}}$.
Intuitively, $(\nu, s, \nu') \in \Tr$ denotes that the state $\nu$ transits to 
the state $\nu'$ after executing the statement $s$.
\begin{eqnarray*}
  \Prog & ::= & \epsilon \ |\ \Stmt; \Prog
\end{eqnarray*}
A \emph{program} is a sequence of statements. We denote the empty program by 
$\epsilon$. Observe that conditional branches are not allowed in our
domain specific language to prevent timing attacks. The semantics of a
program is defined by the relation 
$\Tr^* \subseteq \St \times \Prog \times \St$ where
$(\nu, \epsilon, \nu) \in \Tr^*$ and
$(\nu, s; p, \nu'') \in \Tr^* \textmd{ if }
    \textmd{there is a } \nu' \textmd{ with }
    (\nu, s, \nu') \in \Tr$ and
    $(\nu, p, \nu'') \in \Tr^*$.
We write $\nu \goesto{p} \nu'$ when $(\nu, p, \nu') \in \Tr^*$.

For algebraic specifications, $\top$ denotes the Boolean value
$\mathit{tt}$. We admit polynomial equations $e = e'$ and modular
polynomial equations $e \equiv e' \mod n$ as atomic predicates.
A \emph{predicate} $q \in \Pred$ is a conjunction of atomic
predicates. For 
$\nu \in \St$, write $\bbfZ \models q[\nu]$ if $q$
evaluates to $\btt$ after replacing each variable $v$ with
$\nu(v)$; $\nu$ is called a \emph{$q$-state}.
\begin{eqnarray*}
  \Pred & ::= & \top
     \ |\   \Expr = \Expr
     \ |\   \Expr \equiv \Expr \mod \Nat
     \ |\   \Pred \wedge \Pred\\
  \Spec & ::= & \cond{\Pred} \Prog \cond{\Pred}
\end{eqnarray*}
We follow Hoare's formalism in algebraic specifications of
mathematical constructs~\cite{H:69:ABCP}. In an algebraic
specification $\hoaretriple{q}{p}{q'}$, $q$ and $q'$ are the \emph{pre}- and
\emph{post-conditions} of $p$ respectively. Given $q, q' \in \Pred$ and $p
\in \Prog$, $\hoaretriple{q}{p}{q'}$ is \emph{valid}
(written $\models \hoaretriple{q}{p}{q'}$) if for every $\nu, \nu' \in
\St$, $\bbfZ \models q[\nu]$ and $\nu \goesto{p} \nu'$ imply
$\bbfZ \models q'[\nu']$. Less formally, $\models
\hoaretriple{q}{p}{q'}$ if executing $p$ from a $q$-state always
results in a $q'$-state. 
\begin{figure}[ht]
  \centering
  \[
  \begin{array}{lclcl}
    \begin{array}{rcl}
    \textmd{1:} && {r}_0 \leftarrow {x}_0; \\
    \textmd{2:} && {r}_1 \leftarrow {x}_1; \\
    \textmd{3:} && {r}_2 \leftarrow {x}_2; \\
    \textmd{4:} && {r}_3 \leftarrow {x}_3; \\
    \textmd{5:} && {r}_5 \leftarrow {x}_4; \\
    \end{array}
    &\hspace{.05\textwidth}&
    \begin{array}{rcl}
    \textmd{6:} && 
      {r}_0 \leftarrow {r}_0 + {4503599627370458}; \\
    \textmd{7:} &&
      {r}_1 \leftarrow {r}_1 + {4503599627370494}; \\
    \textmd{8:} &&
      {r}_2 \leftarrow {r}_2 + {4503599627370494}; \\
    \textmd{9:} &&
      {r}_3 \leftarrow {r}_3 + {4503599627370494}; \\
    \textmd{10:} && 
      {r}_4 \leftarrow {r}_4 + {4503599627370494};\\
    \end{array}
    &\hspace{.05\textwidth}&
    \begin{array}{rcl}
    \textmd{11:} && {r}_0 \leftarrow {r}_0 - {y}_0; \\
    \textmd{12:} && {r}_1 \leftarrow {r}_1 - {y}_1; \\
    \textmd{13:} && {r}_2 \leftarrow {r}_2 - {y}_2; \\
    \textmd{14:} && {r}_3 \leftarrow {r}_3 - {y}_3; \\
    \textmd{15:} && {r}_4 \leftarrow {r}_4 - {y}_4;
    \end{array}
  \end{array}
  \]
  \caption{Subtraction \dslcode{sub}}
%$(\dslcode{x}_0, \dslcode{x}_1, \dslcode{x}_2, \dslcode{x}_3,
%\dslcode{x}_4, \dslcode{y}_0, \dslcode{y}_1, \dslcode{y}_2,
%\dslcode{y}_3, \dslcode{y}_4)$
  \label{figure:dsl:subtraction}
\end{figure}

Figure~\ref{figure:dsl:subtraction} gives a simple yet real
implementation of subtraction over $\bbfGF(\myprime)$. 
In the figure, a number in $\bbfGF(\varrho)$ 
is represented by five 51-bit non-negative integers. The variables
${x}_0, {x}_1, {x}_2, {x}_3, {x}_4$ for instance represent 
$\mathit{radix51}({x}_4, {x}_3, {x}_2, {x}_1, {x}_0) \defn
2^{51 \times 4} {x}_4 + 2^{51 \times 3} {x}_3 + 
2^{51 \times 2} {x}_2 + 2^{51 \times 1} {x}_1 + 
2^{51 \times 0} {x}_0$. The result of
subtraction is stored in the variables ${r}_0, {r}_1, {r}_2, {r}_3, {r}_4$. 
Given $0 \leq {x}_0,$ ${x}_1,$ ${x}_2,$ ${x}_3,$ ${x}_4,$ ${y}_0,$ 
${y}_1,$ ${y}_2,$ ${y}_3,$ ${y}_4 < 2^{51}$, 
the algebraic specification of the mathematical construct is therefore
\[
\begin{array}{rcl}
\cond{\top} &
\dslcode{sub} &
\cond{
   \begin{array}{c}
     \mathit{radix51}({x}_4, {x}_3, {x}_2, {x}_1, {x}_0) -
     \mathit{radix51}({y}_4, {y}_3, {y}_2, {y}_1, {y}_0)\\
     \equiv 
     \mathit{radix51}({r}_4, {r}_3, {r}_2, {r}_1, {r}_0)
     \mod \myprime
   \end{array}
}.
\end{array}
\]

Note that the variables ${r}_i$'s are added with constants
after they are initialized with ${x}_i$'s but before
${y}_i$'s are subtracted from them. It is not hard to see that
$2\myprime = \mathit{radix51}$ $(4503599627370494,$ $4503599627370494,$
$4503599627370494,$ $4503599627370494,$ $4503599627370458)$
after tedious computation. Hence $\mathit{radix51}({r}_4,$
${r}_3,$ ${r}_2,$ ${r}_1,$ ${r}_0)$ $=$
$\mathit{radix51}({x}_4,$ ${x}_3,$ ${x}_2,$ ${x}_1,$ ${x}_0)$ $+$ 
$2 \myprime $ $-$
$\mathit{radix51}({y}_4,$ ${y}_3,$ ${y}_2,$ ${y}_1,$ ${y}_0)$ 
$\equiv $
$\mathit{radix51}({x}_4,$ ${x}_3,$ ${x}_2,$ ${x}_1,$ ${x}_0)$ $-$
$\mathit{radix51}({y}_4,$ ${y}_3,$ ${y}_2,$ ${y}_1,$ ${y}_0)$ $\mod 
\ \myprime$. The program in
Figure~\ref{figure:dsl:subtraction} is correct. 
Characteristics of large Galois fields are
regularly exploited in mathematical constructs for correctness and
efficiency. Our domain specific language can easily model such
specialized programming techniques. Indeed,
the reason for
adding constants is to prevent underflow. If the constants were not
added, the subtraction in lines~11 to 15 could give negative and hence
%
incorrect results. {Please note that ranges are a complicated
  issue. The subtraction in Figure~\ref{figure:dsl:subtraction} gives results that are correct
  but possibly {overflowing} ($>\varrho$), which can and must be
  accounted for later.} 

%%% Local Variables: 
%%% mode: latex
%%% eval: (TeX-PDF-mode 1)
%%% eval: (TeX-source-correlate-mode 1)
%%% TeX-master: "certified_vcg"
%%% End: 
