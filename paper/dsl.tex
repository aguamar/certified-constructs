
One of the big issues with modern cryptography is how the assumptions
match up with reality. In many situations, unexpected channels
through which information can leak to the attacker can cause the
cryptosystem to be broken. Typically this is about timing, or
electric power used. Implementations which are not defeated through
such means are called ``side-channel resilient'' implementations.
Clearly in such implementations the execution time are kept constant
(as much as possible).

This turns out in some cases to be harder than one would imagine.
This is chiefly because modern processors have caches and out of order
execution and multitasking. This makes it possible for one execution
thread, even when no privilege is conferred ,to affect the running
time of another -- simply by caching a sufficient amount of its own
data in correct locations through repeated accesses, and then
observing the running time of the other thread. The instructions in
the other thread which uses the ``evicted'' data (to make room for the
data of the eavesdropping thread) then has to take more time getting
their data back to the cache. Such attacks are quite practical.
\todo{insert reference to cache-timing attacks.}

Thus, the innocuous actions of (a) executing the CPU's conditional
branch instruction dependent on a secret bit, and (b) executing an
indirect load instruction using a secret value in the register as the
address, are both potentially dangerous leaks of information.

As such, modern cryptographic implementations often have to resort to
seemingly contrived sequences of conditional moves and arithmetic
manipulations to achieve the same result as simply using the CPU
instructions designed for the purpose. This of course slows the
computation down and waste resources but is inevitable because
secret-dependent conditional actions and table lookups are actually
very useful tasks that sometimes simply must be performed.

There are some side results of the above, one is that we are not often
faced with secret-dependent branching or table-lookups in the assembly
instructions, but a language describing cryptographic code might
include pseudo-instructions to cover instruction sequences, phrases in
the language if you will, that is used to achieve the same effect.

Consider the following syntactic class for expressions in our domain
specific language:

\begin{eqnarray*}
  N & ::= & \dslcode{1}\ |\ \dslcode{2}\ |\ \cdots\\
  Z & ::= & \cdots \ |\ \dslcode{-2}\ |\ \dslcode{-1} \ |\ 0\ |\ 
            \dslcode{1}\ |\ \dslcode{2}\ |\ \cdots\\
  V & ::= & \dslcode{x} \ |\ \dslcode{y} \ |\ \dslcode{z} \ |\ \cdots\\
  E & ::= &  Z \ |\ V \ |\  \dslcode{-}E \ |\ E \dslcode{+} E 
             \ |\ E \dslcode{-} E
             \ |\ E \times E \ |\ \dslcode{Pow} (E, N)
\end{eqnarray*}

We allow exact integers as constants in the domain specific
language. Variables are thus integer variables. An expression can be a
constant, a variable, or a negative expression. Addition, subtraction,
and multiplication of expressions are available. The expression
$\dslcode{Pow}(e, n)$ denotes $e^n$ for any expression $e$ and positive
integer $n$. More formally, let $\St \defn V \rightarrow
\bbfZ$ and $\nu \in \St$ be a \emph{state}. That is,
a {state} $\nu$ is a mapping from variables to integers. Define the
semantic function $\valueof{e}_{\nu}$ for an expression $e$ as follows.
\[
\begin{array}{rclcrcl}
  \valueof{{i}}_{\nu} & \defn & i \textmd{  for ${i} \in Z$} 
  & \hspace{.1\textwidth} &
  \valueof{{v}}_{\nu} & \defn & \nu({v}) 
     \textmd{  for ${v} \in V$}\\
  \valueof{-e}_{\nu} & \defn & -_{\bbfZ}\valueof{e}_{\nu}
  &&
  \valueof{e_0 + e_1}_{\nu} & \defn & 
     \valueof{e_0}_{\nu} +_{\bbfZ} \valueof{e_1}_{\nu} \\
  \valueof{e_0 - e_1}_{\nu} & \defn & 
     \valueof{e_0}_{\nu} -_{\bbfZ} \valueof{e_1}_{\nu}
  &&
  \valueof{e_0 \times e_1}_{\nu} & \defn & 
     \valueof{e_0}_{\nu} \times_{\bbfZ} \valueof{e_1}_{\nu} \\
  \valueof{\dslcode{Pow}(e, n)}_{\nu} & \defn & 
     (\valueof{e}_{\nu})^{\valueof{n}_{\nu}}
\end{array}
\]

Note that there is no expression for quotients, bitwise logical
operations. Bitwise left shifting however can be modeled by
multiplying $\dslcode{Pow}(2, n)$. Although \mydsl models a (very) 
small subset of assembly, it suffices to encode crucial low-level
mathematical constructs in X25519.

\begin{eqnarray*}
  S & ::= & V \leftarrow E 
            \ |\  \concat{V}{V} \leftarrow \dslcode{Split} (E, N)\\
  P & ::= & \epsilon \ |\ S; P
\end{eqnarray*}

For low-level mathematical constructs, only assignments are allowed.
The statement $v \leftarrow e$ assigns the value of $e$
to the variable $v$. The statement $\concat{v_h}{v_l} \leftarrow
\dslcode{Split}(e, n)$ splits the value of $e$ into higher and lowest
$n$ bits. The higher and lower bits are stored in $v_h$ and $v_l$
respectively. More formally, define the relation $\Tr \subseteq \St
\times S \times \St$ as follows.
\begin{eqnarray*}
  (\nu, v \leftarrow e, \nu') \in \Tr & \textmd{if} &
  \nu' = \nu[v \mapsto \valueof{e}_{\nu}]\\
  (\nu, \concat{v_h}{v_l} \leftarrow \dslcode{Split} (e, n), \nu') \in
  \Tr & \textmd{if} &
  \nu' = \nu[v_h \mapsto h, v_l \mapsto l]
\end{eqnarray*}
where
$l = \valueof{e}_{\nu} \mod 2^{\valueof{n}_{\nu}}$ and
$h = (\valueof{e}_{\nu} - l) \div 2^{\valueof{n}_{\nu}}$.


A program is a sequence of statements. We denote the
empty program by $\epsilon$. Observe that conditional branches are not
allowed in cryptographic programs to prevent timing attacks. Every
loop must repeat a fixed number of times and hence can be unfolded
into a sequence of assignments.

For specifications, our domain specific language allows arbitrary
conjunctions of (modulo) equations over expressions. The notation
$\top$ denotes the Boolean value $\mathit{tt}$. We follow Hoare's
formalism in cryptographic program specifications~\cite{H:69:ABCP}. 
For any $q \in Q$, a state $\nu$ is a \emph{$q$-state} if 
$q$ evaluates to $\mathit{tt}$ after replacing each variable $x$ with
$\nu (x)$. Given $q, q' \in Q$ and $p \in P$, $\models
\hoaretriple{q}{p}{q'}$ 
denotes that the program $p$ will terminate and end in a $q'$-state
provided it starts from any $q$-state. In $\hoaretriple{q}{p}{q'}$,
$q$ and $q'$ are the pre- and post-conditions of $p$ respectively.

\begin{eqnarray*}
  Q & ::= & \top
     \ |\   E = E
     \ |\   E \equiv E \mod N
     \ |\   Q \wedge Q\\
  H & ::= & \cond{Q} P \cond{Q}
\end{eqnarray*}


\begin{figure}[ht]
  \centering
  \[
  \begin{array}{lclcl}
    \begin{array}{rcl}
    \textmd{1:} && \dslcode{r}_0 \leftarrow \dslcode{x}_0; \\
    \textmd{2:} && \dslcode{r}_1 \leftarrow \dslcode{x}_1; \\
    \textmd{3:} && \dslcode{r}_2 \leftarrow \dslcode{x}_2; \\
    \textmd{4:} && \dslcode{r}_3 \leftarrow \dslcode{x}_3; \\
    \textmd{5:} && \dslcode{r}_5 \leftarrow \dslcode{x}_4; \\
    \end{array}
    &\hspace{.05\textwidth}&
    \begin{array}{rcl}
    \textmd{6:} && 
      \dslcode{r}_0 \leftarrow \dslcode{r}_0 + \dslcode{4503599627370458}; \\
    \textmd{7:} &&
      \dslcode{r}_1 \leftarrow \dslcode{r}_1 + \dslcode{4503599627370494}; \\
    \textmd{8:} &&
      \dslcode{r}_2 \leftarrow \dslcode{r}_2 + \dslcode{4503599627370494}; \\
    \textmd{9:} &&
      \dslcode{r}_3 \leftarrow \dslcode{r}_3 + \dslcode{4503599627370494}; \\
    \textmd{10:} && 
      \dslcode{r}_4 \leftarrow \dslcode{r}_4 + \dslcode{4503599627370494};\\
    \end{array}
    &\hspace{.05\textwidth}&
    \begin{array}{rcl}
    \textmd{11:} && \dslcode{r}_0 \leftarrow \dslcode{r}_0 - \dslcode{y}_0; \\
    \textmd{12:} && \dslcode{r}_1 \leftarrow \dslcode{r}_1 - \dslcode{y}_1; \\
    \textmd{13:} && \dslcode{r}_2 \leftarrow \dslcode{r}_2 - \dslcode{y}_2; \\
    \textmd{14:} && \dslcode{r}_3 \leftarrow \dslcode{r}_3 - \dslcode{y}_3; \\
    \textmd{15:} && \dslcode{r}_4 \leftarrow \dslcode{r}_4 - \dslcode{y}_4;
    \end{array}
  \end{array}
  \]
  \caption{Subtraction \dslcode{sub}}
%$(\dslcode{x}_0, \dslcode{x}_1, \dslcode{x}_2, \dslcode{x}_3,
%\dslcode{x}_4, \dslcode{y}_0, \dslcode{y}_1, \dslcode{y}_2,
%\dslcode{y}_3, \dslcode{y}_4)$
  \label{figure:dsl:subtraction}
\end{figure}

Figure~\ref{figure:dsl:subtraction} gives a simple yet real
implementation of subtraction over $\bbfGF(\myprime)$. 
In the figure, a number in $\bbfGF(\varrho)$ 
is represented by five 51-bit unsigned integers. The variables
$\dslcode{x}_0, \dslcode{x}_1, \dslcode{x}_2, \dslcode{x}_3,
\dslcode{x}_4$ for instance represent 
$\mathit{radix51}(\dslcode{x}_4, \dslcode{x}_3, \dslcode{x}_2,
\dslcode{x}_1, \dslcode{x}_0) \defn
2^{51 \times 4} \dslcode{x}_4 + 2^{51 \times 3} \dslcode{x}_3 + 
2^{51 \times 2} \dslcode{x}_2 + 2^{51 \times 1} \dslcode{x}_1 + 
2^{51 \times 0} \dslcode{x}_0$. The result of
subtraction is stored in the variables $\dslcode{r}_0, \dslcode{r}_1,
\dslcode{r}_2, \dslcode{r}_3, \dslcode{r}_4$. 
Given $0 \leq \dslcode{x}_0,$ $\dslcode{x}_1,$ $\dslcode{x}_2,$
$\dslcode{x}_3,$ $\dslcode{x}_4,$ $\dslcode{y}_0,$ $\dslcode{y}_1,$
$\dslcode{y}_2,$ $\dslcode{y}_3,$ $\dslcode{y}_4 < 2^{51}$, 
the specification of the program is therefore
\[
\begin{array}{c}
\cond{\top}\\
\dslcode{sub}\\
%(\dslcode{x}_0, \dslcode{x}_1, \dslcode{x}_2, \dslcode{x}_3,
%\dslcode{x}_4, \dslcode{y}_0, \dslcode{y}_1, \dslcode{y}_2,
%\dslcode{y}_3, \dslcode{y}_4)\\
\cond{\mathit{radix}(\dslcode{x}_4, \dslcode{x}_3, \dslcode{x}_2,
\dslcode{x}_1, \dslcode{x}_0) -
\mathit{radix51}(\dslcode{y}_4, \dslcode{y}_3, \dslcode{y}_2,
\dslcode{y}_1, \dslcode{y}_0)
\equiv
\mathit{radix51}(\dslcode{r}_4, \dslcode{r}_3, \dslcode{r}_2,
\dslcode{r}_1, \dslcode{r}_0)
\mod \myprime
}.
\end{array}
\]

Note that the variables $\dslcode{r}_i$'s are added with constants
after they are initizlied with $\dslcode{x}_i$'s but before
$\dslcode{y}_i$'s are subtracted from them. It is not hard to see that
$2\myprime = \mathit{radix51} (4503599627370458,$ $4503599627370494,$
$4503599627370494,$ $4503599627370494,$ $4503599627370494)$
after tedious computation. Hence $\mathit{radix51}(\dslcode{r}_4,$
$\dslcode{r}_3,$ $\dslcode{r}_2,$ $\dslcode{r}_1,$ $\dslcode{r}_0)$ $=$
$\mathit{radix51}(\dslcode{x}_4,$ $\dslcode{x}_3,$ $\dslcode{x}_2,$
$\dslcode{x}_1,$ $\dslcode{x}_0)$ $+$ $2 \myprime $ $-$
$\mathit{radix51}(\dslcode{y}_4,$ $\dslcode{y}_3,$ $\dslcode{y}_2,$
$\dslcode{y}_1,$ $\dslcode{y}_0)$ $\equiv $
$\mathit{radix51}(\dslcode{x}_4,$ $\dslcode{x}_3,$ $\dslcode{x}_2,$
$\dslcode{x}_1,$ $\dslcode{x}_0)$ $-$
$\mathit{radix51}(\dslcode{y}_4,$ $\dslcode{y}_3,$ $\dslcode{y}_2,$
$\dslcode{y}_1,$ $\dslcode{y}_0)$ $\mod \myprime $. The program in
Figure~\ref{figure:dsl:subtraction} is correct. The reason for
adding constants is to prevent underflow. If the constants were not
added, the subtraction in lines~11 to 15 could give negative and hence
incorrect results. Indeed, characteristics of large Galois fields are
regularly exploited in cryptographic programs for correctness and
efficiency. Our domain specific language can easily model such
specialized programming techniques and is most suitable for low-level
cryptographic programs.
