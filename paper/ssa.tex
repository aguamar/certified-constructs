
A program is in \emph{single static assignment} form if every variable
is assigned at most once. \todo{reference?} 
Our next task is to correctly transform any
program specification $\hoaretriple{q}{p}{q'}$ to single static
assignments for any $q, q' \in Q$ and $p \in P$. Our transformation
maintains a finite mapping $\theta$ from program variables to
non-negative integers. For any program variable $v$, $v^{\theta(v)}$ is
the most recently assigned program variable corresponding to $v$. 
Such variables are always referred in
expressions. Algorithm~\ref{algorithm:ssa-expressions} shows how
to transform expressions with the finite mapping $\theta$ by
structural induction.

\begin{algorithm}
  \begin{algorithmic}[1]
    \Function{SSAE}{$\theta$, $e$}
    \Match{$e$}
      \Case{$V$} \Return $e^{\theta(e)}$ \EndCase
      \Case{$Z$} \Return $e$ \EndCase
      \Case{$-e'$} \Return $-$\Call{SSAE}{$\theta$, $e'$} \EndCase
      \Case{$e' + e''$} 
        \Return \Call{SSAE}{$\theta$, $e'$} $+$ 
                \Call{SSAE}{$\theta$, $e''$}
      \EndCase
      \Case{$e' - e''$} 
        \Return \Call{SSAE}{$\theta$, $e'$} $-$ 
                \Call{SSAE}{$\theta$, $e''$}
      \EndCase
      \Case{$e' \times e''$} 
        \Return \Call{SSAE}{$\theta$, $e'$} $\times$ 
                \Call{SSAE}{$\theta$, $e''$}
      \EndCase
      \Case{$\dslcode{Pow}$($e'$, $n$)}
        \Return $\dslcode{Pow}$(\Call{SSAE}{$\theta$, $e'$}, $n$)
      \EndCase
    \EndMatch
    \EndFunction
  \end{algorithmic}
  \caption{Single Static Assignment Transformation for Expressions}
  \label{algorithm:ssa-expressions}
\end{algorithm}

Similarly, predicates always refer to most recently assigned
variables. They are transformed by the finite mapping $\theta$
accordingly (Algorithm~\ref{algorithm:ssa-predicates}). Thanks to the
formalization of finite mappings in \coq. Both
Algorithms~\ref{algorithm:ssa-expressions}
and~\ref{algorithm:ssa-predicates} are easily specified in the proof
assistant.

\begin{algorithm}
  \begin{algorithmic}[1]
    \Function{SSAQ}{$\theta$, $q$}
    \Match{$q$}
      \Case{$\top$} \Return $\top$ \EndCase
      \Case{$e' = e''$} 
        \Return \Call{SSAE}{$\theta$, $e$} = \Call{SSAE}{$\theta$, $e'$}
      \EndCase
      \Case{$e' \equiv e'' \mod n$} 
        \Return \Call{SSAE}{$\theta$, $e$} $\equiv$ 
                \Call{SSAE}{$\theta$, $e'$} $\mod n$
      \EndCase
      \Case{$q' \wedge q''$}
        \Return \Call{SSAQ}{$\theta$, $q'$} $\wedge$
                \Call{SSAQ}{$\theta$, $q''$}
      \EndCase
    \EndMatch
    \EndFunction
  \end{algorithmic}
  \caption{Single Static Assignment Transformation for Predicates}
  \label{algorithm:ssa-predicates}
\end{algorithm}

Statement transformation is slightly more complicated. For
expressions on the right hand side, they are transformed by the given
finite mapping $\theta$. Algorithm~\ref{algorithm:ssa-statements} then
updates the mapping and renames variables on the left hand side of
statements. In the algorithm, $\theta\mymapsto{v}{n}$
denotes the finite mapping which simultaneously maps $v$ to
$n$ and $u$ to $\theta(u)$ for every $u \neq v$.

\begin{algorithm}
  \begin{algorithmic}[1]
    \Function{SSAStatement}{$\theta$, $s$}
    \Match{$s$}
      \Case{$v \leftarrow e$}
        \State{$\theta' \leftarrow \theta\mymapsto{v}{\theta(v) + 1}$}
        \State{\Return $\langle \theta'$, 
                $v^{\theta'(v)} \leftarrow$ \Call{SSAE}{$\theta$, $e$}$\rangle$}
      \EndCase
      \Case{$\concat{v_h}{v_l} \leftarrow \dslcode{Split}(e, n)$}
        \State{$\theta_h \leftarrow \theta\mymapsto{v_h}{\theta(v_h) + 1}$}
        \State{$\theta_l \leftarrow \theta_h\mymapsto{v_l}{\theta_h(v_l) + 1}$}
        \State{\Return $\langle \theta_l$,
                $\concat{v_h^{\theta_h(v_h)}}{v_l^{\theta_l(v_l)}} \leftarrow
                \dslcode{Split}($\Call{SSAE}{$\theta$, $e$}$, n)
                \rangle$
              }
      \EndCase
    \EndMatch
    \EndFunction
  \end{algorithmic}
  \caption{Single Static Assignement Transformation for Statements}
  \label{algorithm:ssa-statements}
\end{algorithm}

It is now straightforward to transform programs to single static
assignment form. Using an initial mapping from variables to integers,
the transformation starts from the first statement and obtains an
updated mapping with a statement in single static assignment form. It
continues to transform the next statement with the updated mapping
(Algorithm~\ref{algorithm:ssa-programs}). 

\begin{algorithm}
  \begin{algorithmic}[1]
    \Function{SSAProgram}{$\theta$, $p$}
    \Match{$p$}
      \Case{$\epsilon$}
        \Return $\langle \theta,$ $\epsilon \rangle$
      \EndCase
      \Case{$s; pp$}
        \State{$\langle \theta'$, $s' \rangle$ $\leftarrow$ 
                 \Call{SSAStatement}{$\theta$, $s$}}
        \State{$\langle \theta''$, $pp'' \rangle$ $\leftarrow$ 
                 \Call{SSAProgram}{$\theta'$, $pp$}}
        \State{\Return $\langle \theta''$, $s'; pp'' \rangle$}
      \EndCase
    \EndMatch
    \EndFunction
  \end{algorithmic}
  \caption{Single Static Assignment for Programs}
  \label{algorithm:ssa-programs}
\end{algorithm}



With the \coq specifications, we establish the correctness of our
algorithms in two steps. We first show that the transformation indeed
produces a program in single static assignment form.
\begin{theorem}
  Let $\theta_0(v) = 0$ for every $v \in V$ and $p \in P$.
  If $\langle \hat{\theta}, \hat{p} \rangle$ $=
  \textsc{SSAProgram}(\theta_0, p)$, then 
  $\hat{p}$ is in single static assignment form.
\end{theorem}

Without loss of generality, we assume programs are well-formed. 
A program is \emph{well-formed} if
\begin{itemize}
\item for every $\concat{v_h}{v_l} \leftarrow \dslcode{Split}(e, n)$, $v_h
  \neq v_l$; and
\item every variable must be assigned to a value before being used.
\end{itemize}
Well-formedness is preserved by our transformation. Formally, we have
the following theorem.
\begin{theorem}
  Let $\theta_0(v) = 0$ for every $v \in V$ and $p \in P$ a
  well-formed program. If $\langle \hat{\theta}, \hat{p} \rangle$ $=$ 
  $\textsc{SSAProgram}(\theta_0, p)$, then $\hat{p}$ is well-formed.
\end{theorem}

The next theorem states that our transformation is both sound and
complete. That is, a program specification is valid if and only if its
corresponding specification in single static assignment form is valid.
All theorems are formally certified in \coq. 
\begin{theorem}
  Let $\theta_0(v) = 0$ for every $v \in V$. For every $q, q' \in Q$
  and $p \in P$,
  \begin{center}
    $\models \hoaretriple{q}{p}{q'}$ if and only if
    $\models \hoaretriple{\textsc{SSAQ}(\theta_0, q)}
    {\hat{p}}
    {\textsc{SSAQ}(\hat{\theta}, q')}$
  \end{center}
  where $\langle \hat{\theta}, \hat{p} \rangle =
  \textsc{SSAProgram}(\theta_0, p)$.
  \label{theorem:ssa}
\end{theorem}

\begin{figure}
  \centering
  \[
  \begin{array}{lclcl}
    \begin{array}{rcl}
    \textmd{1:} && \dslcode{r}^0_0 \leftarrow \dslcode{x}^0_0; \\
    \textmd{2:} && \dslcode{r}^0_1 \leftarrow \dslcode{x}^0_1; \\
    \textmd{3:} && \dslcode{r}^0_2 \leftarrow \dslcode{x}^0_2; \\
    \textmd{4:} && \dslcode{r}^0_3 \leftarrow \dslcode{x}^0_3; \\
    \textmd{5:} && \dslcode{r}^0_4 \leftarrow \dslcode{x}^0_4; \\
    \end{array}
    &\hspace{.05\textwidth}&
    \begin{array}{rcl}
    \textmd{6:} && 
      \dslcode{r}^1_0 \leftarrow \dslcode{r}^0_0 + \dslcode{4503599627370458}; \\
    \textmd{7:} &&
      \dslcode{r}^1_1 \leftarrow \dslcode{r}^0_1 + \dslcode{4503599627370494}; \\
    \textmd{8:} &&
      \dslcode{r}^1_2 \leftarrow \dslcode{r}^0_2 + \dslcode{4503599627370494}; \\
    \textmd{9:} &&
      \dslcode{r}^1_3 \leftarrow \dslcode{r}^0_3 + \dslcode{4503599627370494}; \\
    \textmd{10:} && 
      \dslcode{r}^1_4 \leftarrow \dslcode{r}^0_4 + \dslcode{4503599627370494};\\
    \end{array}
    &\hspace{.05\textwidth}&
    \begin{array}{rcl}
    \textmd{11:} && \dslcode{r}^2_0 \leftarrow \dslcode{r}^1_0 - \dslcode{y}^0_0; \\
    \textmd{12:} && \dslcode{r}^2_1 \leftarrow \dslcode{r}^1_1 - \dslcode{y}^0_1; \\
    \textmd{13:} && \dslcode{r}^2_2 \leftarrow \dslcode{r}^1_2 - \dslcode{y}^0_2; \\
    \textmd{14:} && \dslcode{r}^2_3 \leftarrow \dslcode{r}^1_3 - \dslcode{y}^0_3; \\
    \textmd{15:} && \dslcode{r}^2_4 \leftarrow \dslcode{r}^1_4 - \dslcode{y}^0_4;
    \end{array}
  \end{array}
  \]
  \caption{Subtraction \dslcode{subSSA} in Single Static Assignment Form}
% $(\dslcode{x}_0, \dslcode{x}_1, \dslcode{x}_2, \dslcode{x}_3,
% \dslcode{x}_4, \dslcode{y}_0, \dslcode{y}_1, \dslcode{y}_2,
% \dslcode{y}_3, \dslcode{y}_4)$
  \label{figure:translation:subtraction-ssa}
\end{figure}

\noindent
\emph{Example.}
Figure~\ref{figure:translation:subtraction-ssa} gives the subtraction program
\textsc{sub} in single static assignment form. Starting from $0$, the
index of a variable is incremented when the variable is assigned to an
expression. After the single static assignment translation, the
variables $\dslcode{x}_i$, $\dslcode{y}_i$ are indexed by $0$ and
$\dslcode{r}_i$ are indexed by $2$ for $0 \leq i \leq 4$. 
Subsequently, variables in pre- and post-conditions of the algebraic
specification for subtraction needs to be re-indexed. The
corresponding algebraic specification is as follows.
\[
\begin{array}{c}
\cond{\top}\\
\dslcode{subSSA}\\
% (\dslcode{x}_0, \dslcode{x}_1, \dslcode{x}_2, \dslcode{x}_3,
% \dslcode{x}_4, \dslcode{y}_0, \dslcode{y}_1, \dslcode{y}_2,
% \dslcode{y}_3, \dslcode{y}_4)\\
\cond{\mathit{radix}(\dslcode{x}^0_4, \dslcode{x}^0_3, \dslcode{x}^0_2,
\dslcode{x}^0_1, \dslcode{x}^0_0) -
\mathit{radix51}(\dslcode{y}^0_4, \dslcode{y}^0_3, \dslcode{y}^0_2,
\dslcode{y}^0_1, \dslcode{y}^0_0)
\equiv
\mathit{radix51}(\dslcode{r}^2_4, \dslcode{r}^2_3, \dslcode{r}^2_2,
\dslcode{r}^2_1, \dslcode{r}^2_0)
\mod \myprime
}.
\end{array}
\]

