
Consider the problem of checking $\models \hoaretriple{q}{p}{q'}$ for
arbitrary $q, q' \in \Pred$ and $p \in \Prog$. Since $q'$ is arbitrary, the
program $p$ may contain fragments irrelevant to $q'$. Program slicing
is a simple yet effective technique to improve efficiency of
verification through simplifying programs~\cite{W:81:PS}.

Here we slice programs backwardly. We initialize the set
of \emph{cared} variables to be the variables appeared in the
post-condition $q'$. Starting from the last statement, we check if
it assigns any cared variables. If so, we keep the
statement and update the cared variables; otherwise, the
statement is skipped. Then we continue to the previous statement.
This process repeats until all statements of the
program are examined.

Our program slicing algorithm requires several auxiliary functions. 
Algorithm~\ref{algorithm:variables-in-expressions} shows how to
compute variables in an expression by structural
induction. If the given expression is an integer,
the empty set is returned; if it is a variable, the singleton with the
variable is returned. Variables in other expressions are computed
recursively. 

\begin{algorithm}
  \begin{algorithmic}[1]
    \Function{VarsInExpr}{$e$}
    \Match{$e$}
      \Case{$i$}
        \Return $\emptyset$
      \EndCase
      \Case{$v$}
        \Return $\{ v \}$
      \EndCase
      \Case{$-e'$}
        \Return \Call{VarsInExpr}{$e'$}
      \EndCase
      \Case{$e' + e''$}
        \Return \Call{VarsInExpr}{$e'$} $\cup$ \Call{VarsInExpr}{$e''$}
      \EndCase
      \Case{$e' - e''$}
        \Return \Call{VarsInExpr}{$e'$} $\cup$ \Call{VarsInExpr}{$e''$}
      \EndCase
      \Case{$e' \times e''$}
        \Return \Call{VarsInExpr}{$e'$} $\cup$ \Call{VarsInExpr}{$e''$}
      \EndCase
      \Case{$\dslcode{Pow}(e', \uscore)$}
        \Return \Call{VarsInExpr}{$e'$}
      \EndCase
    \EndMatch
    \EndFunction
  \end{algorithmic}
  \caption{Variables Occurred in Expressions}
  \label{algorithm:variables-in-expressions}
\end{algorithm}

\vspace{-1em}
\begin{algorithm}
\begin{algorithmic}[1]
  \Function{VarsInPred}{$q$}
  \Match{$q$}
    \Case{$\top$}
      \Return $\emptyset$
    \EndCase
    \Case{$e' = e''$}
      \Return \Call{VarsInExpr}{$e'$} $\cup$ \Call{VarsInExpr}{$e''$}
    \EndCase
    \Case{$e' \equiv e'' \mod \uscore$}
      \Return \Call{VarsInExpr}{$e'$} $\cup$ \Call{VarsInExpr}{$e''$}
    \EndCase
    \Case{$q' \wedge q''$}
      \Return \Call{VarsInPred}{$q'$} $\cup$ \Call{VarsInPred}{$q''$}
    \EndCase
  \EndMatch
  \EndFunction
\end{algorithmic}
\caption{Variables Occurred in Predicates}
\label{algorithm:variables-in-predicates}
\end{algorithm}

Similarly, Algorithm~\ref{algorithm:variables-in-predicates} computes
the variables in a predicate. Using the \coq specification
language \gallina, both algorithms are specified in the proof
assistant.

\begin{algorithm}
  \begin{algorithmic}[1]
    \Function{SliceStmt}{$\mathit{vars}$, $s$}
    \Match{$s$}
      \Case{$v \leftarrow e$}
        \IfThenElse{$v \in \mathit{vars}$}
          {\Return $\langle$\Call{VarsInExpr}{$e$} 
            $\cup$ $(\mathit{vars} \setminus \{ v \})$, $s \rangle$}
          {\Return $\mathit{vars}$}
      \EndCase
      \Case{$\concat{v_h}{v_l} \leftarrow \dslcode{Split} (e, \uscore)$}
        \IfThenElse{$v_h$ or $v_l \in \mathit{vars}$}
          {\Return $\langle$\Call{VarsInExpr}{$e$} $\cup$ 
            $(\mathit{vars} \setminus \{ v_h, v_l \})$, $s \rangle$}
          {\Return $\mathit{vars}$}
      \EndCase
    \EndMatch
    \EndFunction
  \end{algorithmic}
  \caption{Slicing Statements}
  \label{algorithm:slicing-statements}
\end{algorithm}

To slice a statement, we check if the assigned variables are
cared variables. If they are not, we leave the cared
variables intact. 
If the assigned variables are cared variables, we update the cared
variables by excluding the assigned variables but including
the variables in the
expression~(Algorithm~\ref{algorithm:slicing-statements}). In the
algorithm,  
the parameter $\mathit{vars}$ denotes the set of cared variables.
If the given statement assigns to a cared variable, it is
returned with the updated cared variables. Otherwise, only
cared variables are returned.

To slice a program, our algorithm proceeds from the last statement
(Algorithm~\ref{algorithm:slicing-programs}). It invokes
\textsc{SliceStmt} to see if the current statement is relevant
and update cared variables. The algorithm recurses with
updated cared variables and remaining statements.

\begin{algorithm}
  \begin{algorithmic}[1]
    \Function{SliceProg}{$\mathit{vars}$, $p$}
    \Match{$p$}
      \Case{$\epsilon$}
        \Return $\epsilon$
      \EndCase
      \Case{$pp; s$}
        \Match{\Call{SliceStmt}{$\mathit{vars}$, $s$}}
          \Case{$\mathit{vars}'$}
            \Return \Call{SliceProg}{$\mathit{vars}'$, $pp$}
          \EndCase
          \Case{$\langle \mathit{vars}'$, $s' \rangle$}
            \Return \Call{SliceProg}{$\mathit{vars}'$, $pp$}$;\ s'$
          \EndCase
        \EndMatch
      \EndCase
    \EndMatch
    \EndFunction
  \end{algorithmic}
  \caption{Slicing Programs}
  \label{algorithm:slicing-programs}
\end{algorithm}

To avoid ambiguity, we consider \emph{well-formed} programs where
\begin{itemize}
\item for every statement $\concat{v_h}{v_l} \leftarrow
  \dslcode{Split}(e, n)$ in the program, $v_h \neq v_l$; and
\item every non-input program variable must be assigned to a value
  before being used. 
\end{itemize}
Algorithms~\ref{algorithm:slicing-statements}
and~\ref{algorithm:slicing-programs} are specified in \gallina. Their
properties are also specified and proven in \coq. 
Particularly, well-formedness is preserved by program slicing.
\begin{lemma}
  $\textsc{SliceProg}(\textsc{VarsInPred}(q'), p)$ is well-formed
  for every $q' \in \Pred$ and well-formed program $p \in \Prog$.
  \label{lemma:slicing-well-formed}
\end{lemma}

Moreover, the
soundness and completeness of our program slicing algorithm have been
certified. Formally, we have the following theorem:

\begin{theorem}
  For every $q, q' \in \Pred$ and $p \in \Prog$,
  \begin{center}
  $\models \hoaretriple{q}{p}{q'}$ if and only if
  $\models \hoaretriple{q}{\textsc{SliceProg}(\textsc{VarsInPred}(q'), p)}
  {q'}$.
  \end{center}
  \label{theorem:program-slicing}
\end{theorem}