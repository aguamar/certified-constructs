
We evaluate our techniques in real-world low-level mathematical
constructs of X25519. 
In elliptic curve cryptography, arithmetic computation over
large finite fields is required for cryptographic primitives. For
instance, the elliptic curve $y^2 = x^3 + 486662 x^2 + x$ used in 
Curve25519 is over the Galois field $\bbfF = \bbfGF(\varrho)$ with
$\varrho = 2^{255} - 19$. To
make the finite field $\bbfF$ explicit, we rewrite the elliptic
curve in the following equation: 
\begin{equation}
  \label{eq:curve25519}
  y \Ftimes y \Feq x \Ftimes x \Ftimes x \Fplus
  486662 \Ftimes x \Ftimes x \Fplus x.
\end{equation}

Since arithmetic computation is over the Galois field and every
element of $\bbfGF(\varrho)$ can be represented by a 255-bit number,
any pair $(x, y)$ satisfying (\ref{eq:curve25519}) (called a
\emph{point} on the curve) can be represented by a pair of 255-bit
numbers. It can be shown that points on Curve25519 with the point at
infinity form a commutative group $\bbfG = (G, \Gplus, \Gzero)$
with $G = \{ (x, y) : x, y \textmd{ satisfying } (\ref{eq:curve25519})
\}$. Let $P_0 = (x_0, y_0), P_1 = (x_1, y_1) \in G$. We have $-P_0 =
(x_0, -y_0)$. $P_0 \Gplus P_1$ is defined for three exclusive
cases. When $P_0 \neq \pm P_1$, $P_0 + P_1 = (x, y)$ where
\begin{eqnarray}
  m & = & (y_1 \Fminus y_0) \Fdiv (x_1 \Fminus x_0)\\\nonumber
  x & = & b \Ftimes m \Ftimes m \Fminus a \Fminus x_0 \Fminus x_1\\ \nonumber
  y & = & %m \Ftimes (x_0 \Fminus x) \Fminus y_0 
     %= m \Ftimes (x_1 \Fminus x) \Fminus y_1.
          (2 \Ftimes x_1 \Fplus x_2 \Fplus a )\Ftimes m
          \Fminus b \Ftimes m \Ftimes m \Ftimes m\Fminus y_1
  \label{eq:curve25519-group}
\end{eqnarray}
When $P_0=-P_1$, the sum is the point at infinity; When $P_0= P_1$,
$P_0 + P_1 = (x, y)$ where
\begin{eqnarray}
  \label{eq:curve25519-dbl}
  m &=& (3  \Ftimes x_1 \Ftimes x_1 \Fplus 2\Ftimes a\Ftimes x_1\Fplus 1) 
        \Fdiv (2\Ftimes b\Ftimes y_1) \\ \nonumber
   x &=& b \Ftimes m \Ftimes m \Fminus a \Fminus 2 \Ftimes x_1 \\ \nonumber
   y &=& (3 \Ftimes x1\Fplus a) \Ftimes m \Fminus b \Ftimes m \Ftimes m \Ftimes m\Fminus y_1
\end{eqnarray}
%
The induced group $\bbfG$
plays an important role in elliptic curve cryptography. It is
essential to implement the commutative binary operation $\Gplus$ very
efficiently in practice.

%%% Local Variables: 
%%% mode: latex
%%% eval: (TeX-PDF-mode 1)
%%% eval: (TeX-source-correlate-mode 1)
%%% TeX-master: "certified_vcg"
%%% End: 
